\label{sec:discussion}
From the previous sections, it is clear that modern compilers introduce timing vulnerabilities in constant-time programs.
This is devastating for security since it means that developers cannot reliably write constant-time code.
A solution is required.
In this section, we discuss how the problem can be solved and what the implications different solutions have on security and efficiency.

As shown in Section \ref{sec:gcc-vs-clang}, \texttt{clang} is much more conservative when it comes to inserting conditional branches in constant-time programs.
Instead of inserting variable-time conditional branches, \texttt{clang} uses constant-time conditional moves to implement branching control flow.
From this observation, it might be tempting to conclude that always substituting conditional branches with conditional moves is the solution to the problem.
However, there is more to the story.

\texttt{cmov} instructions severely restrict the out-of-order engine on modern CPUs since \texttt{cmov} instructions increase the data dependency between instructions significantly \citep{intel-optimization-reference}. 
Furthermore, branch predictions are not used for \texttt{cmov} instructions, which can lead to significant performance degradation.
Hence, we have to address this trade-off between security and efficiency.

\subsection{Forcing \texttt{cmov} Instructions When Necessary}
\label{sec:forcing-cmov}
This trade-off between security and efficiency has led multiple researchers to propose solutions that only force \texttt{cmov} instructions to be used on critical data \citep{what-you-c,llvm-issues-blog-post}.
In both studies, the authors propose to add the possibility to specify forced \texttt{cmov} instructions in the source code and restrict the backend of the compiler to not being able to optimize the \texttt{cmov} away for those specific instructions.
This solution has the advantage that it does not affect the performance of non-critical parts of the program.
In fact, \citeauthor{what-you-c} show that their solution only introduces a performance penalty of 1\% and sometimes even increases performance.

However, this solution also has its disadvantages.
First, the solution relies heavily on the developer.
The developer has to know which parts of the program are critical, and also be able to use the language extension correctly.
In turn, it means that security relies on human factors, which are error-prone.
Second, the solution relies on the compiler being configurable to avoid forced \texttt{cmov} instructions being optimized away.
As we discussed in Section \ref{sec:gcc-vs-clang} and \ref{sec:general-optimizations}, \texttt{clang} seems to be quite configurable while \texttt{gcc} does not.
Hence, the solution might not be easily applicable to all compilers.

Another solution is to implement vulnerable parts of the program using a domain-specific language that is designed to be constant-time.
\citeauthor{fact} proposed a domain-specific language, called FaCT, for implementing constant-time cryptographic algorithms \citep{fact}.
FaCT is designed to be interoperable with C, and the compiler ensures constant-time code by leveraging the \texttt{ct-verif} tool that is based on verification of the LLVM IR \cite{verifying-constant-time-llvm}.
\texttt{ct-verif} operates on a formalization of constant-time code and verifies based on a theoretically sound and complete methodology.
However, \citeauthor{verifying-constant-time-llvm} note that the verification process of \texttt{ct-verif} is not guaranteed to identify all timing vulnerabilities due to translation between the LLVM IR and machine code, and due to possible incompleteness of their formalization of constant-time code. 
Some of these issues were experimentally verified.
However, generally \texttt{ct-verif} captures significantly more timing vulnerabilities than other verification alternatives \citep{verifying-constant-time-llvm}.
Additionally, \texttt{ct-verif} is efficient at verifying constant-time code, often being much faster than the compilation step, but occasionally being a few times slower.

\subsection{Bullet-proofing the Compiler}
A more radical approach is to force the compiler to always produce constant-time code.
\citeauthor{verified-constant-time-c-comiler} provides a formally verified constant-time C compiler, built on top of the CompCert verified C compiler \citep{verified-constant-time-c-comiler}. 
The constant-time CompCert compiler guarantees that constant-time source code is compiled into constant-time machine code.
This is achieved by instrumenting the operational semantics of CompCert's IR to be able to capture cryptographic constant-time and proving that cryptographic constant-time is preserved during compilation. 
This has immense advantages for security.
However, the security advantages come at the cost of efficiency. 
The constant-time CompCert compiler is significantly slower than \texttt{gcc}.
Experimental results show an efficiency penalty of well over 100\% in some cases \citep{verified-constant-time-c-comiler}.

\subsection{Testing Solutions}
The final category of solutions we identified is black-box testing solutions like DudeCT \citep{dudect} and our tool OptiFuzz.
Concretely, one can test specific implementations for timing vulnerabilities using these tools.
These tests can even be automated to run as part of a continuous integration pipeline.
Solutions in this category have the obvious advantage that they are simple and require no modeling of hardware behavior or language semantics.
It is arguably the most practical solution since it does not require any changes to the compiler or the language.
It also does not induce any performance penalty.

The testing solutions, however, come with a significant security disadvantage in that they are noisy and not exact.
As we showed in Section \ref{sec:general-optimizations}, OptiFuzz will produce false positives and false negatives.
\citeauthor{dudect} raised similar concerns for DudeCT.
Fuzzing tools provide more of an indication of the presence of timing vulnerabilities than a guarantee.

Finally, we want to mention the possibility to combine the solutions.
OptiFuzz differs from other existing solutions in that it operates on the tool level instead of the program level.
For example, we have used OptiFuzz to conclude that \texttt{gcc} is not suitable for compiling constant-time code, while \texttt{clang} is safer.
Hence OptiFuzz might be tailored to be used in combination with the solutions mentioned in Section \ref{sec:forcing-cmov} to detect security vulnerabilities in these.
This would allow tool developers to maximize the security of their tools while keeping a low performance penalty.

In summary, various solutions exist to the problem of timing vulnerabilities in constant-time programs.
The solutions range from solutions that are very efficient but not very secure to solutions that are very secure but not very efficient.
The correct solution depends on the use case.
But importantly, many developers do not make use of these tools even though they are aware of the problem \citep{developer-survey-timing-attacks}.
Hence, we urge developers to start using these tools since, as we showed, relying on the compiler to produce constant-time code is not a viable solution.