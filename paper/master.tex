\documentclass[10pt]{article}
\usepackage[left=0.8in,right=0.8in,bottom=0.5in,top=0.5in]{geometry}
\geometry{a4paper}
\usepackage[parfill]{parskip}    % Activate to begin paragraphs with an empty line rather than an indent
\usepackage{graphicx}
\usepackage{xcolor}
\usepackage{hyperref}
\usepackage{titling}
\usepackage[small,compact]{titlesec}
\usepackage[toc,page]{appendix}

\usepackage{listings}
\usepackage{pgfplots}
\usepackage{adjustbox}
\usepackage{float}
\usepackage{subcaption}
\usepackage{pgffor}
\usepackage{expl3}
\usepackage{xparse}
\usepackage{xfp}
\usetikzlibrary{fit}
\usetikzlibrary{positioning}


\usepackage{amsmath,amssymb}
\DeclareGraphicsRule{.tif}{png}{.png}{`convert #1 `dirname #1`/`basename #1 .tif`.png}



\newcommand\courseName{Language-Based Security}
\newcommand\courseNameAbbrv{LBS}
\newcommand\courseYear{2023}
\newcommand\courseAndYear{\courseNameAbbrv-\courseYear}
\newcommand\reportKind{Exam Report}

\newcommand\groupNumber[1]{
  \makeatletter
  \def\@courseGroupNumber{#1}
  \makeatother
}

\pretitle{\begin{flushright} \bfseries  \large \courseAndYear: \reportKind \end{flushright}   \begin{flushleft}
\bfseries \LARGE}
\posttitle{\par\end{flushleft}\vskip 0.5em}

\preauthor{
\begin{flushleft}
\large \lineskip 0.5em%
\bfseries
\makeatletter Group \@courseGroupNumber \\ \makeatother
\begin{tabular}[t]{@{}l}}
\postauthor{\end{tabular}\par\end{flushleft}}

\predate{\begin{flushleft}\bfseries \large}
\postdate{\par\end{flushleft}}

\usepackage{listings}
\usepackage{pgfplots}
\usepackage{adjustbox}
\usepackage{float}
\usepackage{subcaption}
\usepackage{pgffor}
\usepackage{expl3}
\usepackage{xparse}
\usepackage{xfp}

\newcommand\todo[1]{\textcolor{red}{TODO: #1}}

\pgfplotsset{compat=1.12}

\definecolor{mGreen}{rgb}{0,0.6,0}
\definecolor{mGray}{rgb}{0.5,0.5,0.5}
\definecolor{mPurple}{rgb}{0.58,0,0.82}
\definecolor{backgroundColour}{rgb}{0.95,0.95,0.92}

\definecolor{firstCol}{HTML}{332288}
\definecolor{secondCol}{HTML}{117733}
\definecolor{thirdCol}{HTML}{44AA99}
\definecolor{fourthCol}{HTML}{88CCEE}
\definecolor{fifthCol}{HTML}{DDCC77}
\definecolor{sixthCol}{HTML}{DD2255}

\lstdefinestyle{defstyle}{
    backgroundcolor=\color{backgroundColour},   
    commentstyle=\color{mGreen},
    keywordstyle=\color{magenta},
    numberstyle=\tiny\color{mGray},
    stringstyle=\color{mPurple},
    basicstyle=\footnotesize,
    breakatwhitespace=false,         
    breaklines=true,                 
    captionpos=b,                    
    keepspaces=true,                 
    numbers=left,                    
    numbersep=5pt,                  
    showspaces=false,                
    showstringspaces=false,
    showtabs=false,                  
    tabsize=2
}

\newsavebox{\mybox}

\ExplSyntaxOn
\cs_new:Npn \__afp_ismember_loop:Nnw #1#2#3,
  {
    \quark_if_recursion_tail_stop_do:nn {#3}
      { \prg_return_false: }
    #1 {#2} {#3}
      { \use_i_delimit_by_q_recursion_stop:nw { \prg_return_true: } }
      { \__afp_ismember_loop:Nnw #1 {#2} }
  }
\prg_new_conditional:Npnn \afp_int_ismember:nn #1#2 { p, T, F, TF }
  {
    \__afp_ismember_loop:Nnw \__afp_int_isequal:nnTF {#1} #2 ,
    \q_recursion_tail , \q_recursion_stop
  }
\prg_new_conditional:Npnn \__afp_int_isequal:nn #1#2 { p, T, F, TF }
  {
    \int_compare:nNnTF {#1} = {#2}
      { \prg_return_true: }
      { \prg_return_false: }
  }
\NewExpandableDocumentCommand { \IncludeExperimentResults } { m m }
  {
    \foreach \x in {1,...,#1} {
        \noindent\bool_if:NF { \afp_int_ismember_p:nn {\x} {#2} } {\input{generated_latex/prog\x.tex}}
    }
  }
\ExplSyntaxOff
\usepackage[authoryear]{natbib}

\title{
  Optimizing Away Security in C
}

\groupNumber{10}
\author{Anders B. Clausen \and Johan T. Degn \and Jonathan S. Eilath}

\begin{document}
\maketitle
\thispagestyle{empty}

\todo{Add short guide to read data plots}
\todo{Make figures smaller and better, and remove [H]}
\todo{Remove section from fuzzing and move to analysis}
\todo{Enable appendix}

\section*{Abstract}
Developers strive to write secure cryptographic code that is resistant to various attacks, including timing attacks.
It is a well-established fact that C compilers can introduce timing vulnerabilities into programs, even when the source code is seemingly secure.
However, to our knowledge, the extent of this problem is quantitatively unknown: 
How bad is the problem?
In this paper, we present OptiFuzz, a tool that can be used to test compilers for timing vulnerabilities.
OptiFuzz has been optimized considering the C language specification, the x86-64 architecture, and state-of-the-art statistical analysis.
This has resulted in a simple tool that can reliably indicate the extent of the real-world issue of timing vulnerabilities introduced by compilers.
Using OptiFuzz, we have generated and compiled 600,000 random C programs and shown that between 0.15\% and 12.63\% of these programs are vulnerable to timing attacks when compiled with \texttt{gcc} depending on the configured level of optimization.
Contrarily, we show that \texttt{clang} introduces timing vulnerabilities much more rarely.
This has led us to pinpoint specific optimizations within \texttt{gcc} that are responsible for introducing timing vulnerabilities, as well as specific expression patterns that are vulnerable to resulting in timing vulnerabilities when compiled.
In light of this result, we have discussed how the issue of timing vulnerabilities introduced by compilers can be mitigated by using language-based security techniques from the literature.

\section{Introduction}
\subsection{Related Work}
\cite{what-you-c}
\subsection{Contributions}
\subsection{Overview}

\section{Preliminaries}
\label{sec:preliminaries}
\subsection{Timing Attacks}
Timing attacks are a class of side-channel attacks that exploit the fact that the execution time of a program can depend on the input.
The history of timing attacks goes back several decades where Kocher showed multiple successful timing attacks on well-known cryptographic algorithms such as Diffie-Hellman and RSA \citep{1996-timing-attacks}.
An example of vulnerable code is shown in Figure \ref{fig:timing-attack-example}.
\begin{figure}[H]
  \begin{lstlisting}[style=defstyle,language=C, xleftmargin=6.8cm, xrightmargin=6.8cm]
int foo(int x) {
  if (x < 100) {
    x *= 2;
    x += 7;
  }
  return x;
} \end{lstlisting} 
  \caption{Example of a program vulnerable to a timing attack. 
  Only by analyzing the execution time of the machine code, an attacker can infer whether the input is less than 100 or not.}
  \label{fig:timing-attack-example}
\end{figure}

\subsection{Optimizing Compilers}
Cryptographers will avoid code like the example in Figure \ref{fig:timing-attack-example} and write constant-time code instead.
Code is constant-time if the execution time of the program is independent of the input.
However, the compiler may introduce timing vulnerabilities through optimizations by adding variable-time branches to the machine code.
The issue arises in the analysis and transformation phases of the compiler as illustrated in Figure \ref{fig:optimizing-compiler-pipeline}.

\begin{figure}[H]
  \centering
  \tikzstyle{box} = [rectangle, minimum width=2.8cm, minimum height=0.7cm, text centered, draw=black]
\tikzstyle{arrow} = [thick,->,>=stealth]

\begin{tikzpicture}
  \node (Source Code) [box] {\small Source Code};
  \node (Lexing) [box, fill=lightgray, right of=Source Code, xshift=2.5cm] {\small Lexing};
  \node (Parsing) [box, fill=lightgray, right of=Lexing, xshift=2.5cm] {\small Parsing};
  \node (AST) [box, right of = Parsing, xshift=2.5cm] {\small AST};
  \node (Semantic Analysis) [box, fill=lightgray, right of=AST, xshift=2.5cm] {\small Semantic Analysis};
  \node (Intermediate Representation) [box, below of=Semantic Analysis, yshift=-1.2cm, xshift=0cm] {\small IR};
  \node (Analysis) [box, fill=lightgray, left of=Intermediate Representation, xshift=-2.5cm] {\small Analysis};
  \node (Transformation) [box, fill=lightgray, left of=Analysis, xshift=-2.5cm] {\small Transformation};
  \node (Code Generation) [box, fill=lightgray, left of=Transformation, xshift=-2.5cm] {\small Code Generation};
  \node (Executable) [box, left of=Code Generation, xshift=-2.5cm] {\small Executable};

  \draw [arrow] (Source Code) -- (Lexing);
  \draw [arrow] (Lexing) -- (Parsing);
  \draw [arrow] (Parsing) -- node [fill=white, text=darkgray, yshift=0.7cm] {\small Syntactically Sound} (AST);
  \draw [arrow] (AST) -- (Semantic Analysis);
  \draw [arrow] (Semantic Analysis) -- node [fill=white, text=darkgray] {\small Semantically Sound} (Intermediate Representation);
  \draw [arrow] (Intermediate Representation) -- (Analysis);
  \draw [arrow] (Analysis) -- (Transformation);
  \draw [arrow, dashed] (Transformation) to [bend left=25] (Intermediate Representation);
  \draw [arrow] (Transformation) -- node [fill=white, text=red, yshift=0.7cm] {\small Optimized} (Code Generation);
  \draw [arrow] (Code Generation) -- (Executable);
\end{tikzpicture}
  \caption{The pipeline of an optimizing compiler. After the transformation phase, the IR is optimized and the compiler may have introduced timing vulnerabilities.}
  \label{fig:optimizing-compiler-pipeline}
\end{figure}

Many different kinds of optimization techniques are carried out by optimizing compilers, some of which can introduce timing vulnerabilities.
Some common optimization techniques like common subexpression elimination and strength reduction have been shown to introduce timing vulnerabilities \citep{optimizations-linked-to-timing-attacks}.
To illustrate this point, we look at how common subexpression elimination can introduce timing vulnerabilities.

\subsubsection{Timing Vulnerabilities Through Common Subexpression Elimination}
\label{sec:cse}
Common Subexpression Elimination is an optimization technique that extracts subexpressions that are common across multiple expressions and replaces them with a single variable.
This optimization technique can introduce timing vulnerabilities since it can decrease the number of instructions executed for a specific branch of the code as illustrated in Figure \ref{fig:common-subexpression-elimination}.
Here the common subexpression \texttt{2 * 3 + 5} is extracted and assigned to the variable \texttt{common}, making the \texttt{else}-branch execute faster than the \texttt{then}-branch.

\begin{figure}[H]
  \centering
     \begin{subfigure}[b]{0.3\textwidth}
        \begin{lstlisting}[style=defstyle, language=C]
int foo(int x, int *arr) {
  if (x == SECRET) {
    x = arr[0] * 3 + 5;
    x += arr[1] * 3 + 5;
    x += arr[2] * 3 + 5;
  } else {
    x = 2 * 3 + 5;
    x += 2 * 3 + 5;
    x += 2 * 3 + 5;
  }
  return x;
} \end{lstlisting} 
         \caption{Original code.}
    \end{subfigure}
    \hspace{1cm}
    \begin{subfigure}[b]{0.3\textwidth}
      \begin{lstlisting}[style=defstyle, language=C]
int foo(int x, int *arr) {
  if (x == SECRET) {
    x = arr[0] * 3 + 5;
    x += arr[1] * 3 + 5;
    x += arr[2] * 3 + 5;
  } else {
    // optimized
    int common = 2 * 3 + 5;
    x = 3 * common;
  }
  return x;
} \end{lstlisting} 
       \caption{Optimized code.}
  \end{subfigure}
  \caption{An example of how the common subexpression elimination can introduce timing vulnerabilities in code.}
  \label{fig:common-subexpression-elimination}
\end{figure}

\section{OptiFuzz}
% I propose "generate, fuzz and analyze" as this conveys our pipeline
We created a tool, OptiFuzz, that can be used to generate, analyze and fuzz random C programs.
The source code for OptiFuzz is available on GitHub\footnote{\url{https://github.com/anbclausen/optifuzz}.}.
The goal of OptiFuzz is to quantify the issue of timing attacks introduced by C compilers with different optimization flags enabled.
The tool works as follows:
\begin{itemize}
  \item OptiFuzz generates random C programs consisting of non-branching arithmetic, logical and comparison operations.
  \item OptiFuzz then compiles the generated C programs with different specified optimization flags enabled and inspects the generated assembly for conditional branching instructions introduced by the compiler.
        If branching is found, the program is flagged.
  \item OptiFuzz then fuzzes the flagged programs with various random inputs to test whether the branching instructions can be exploited to leak information about the input.
  \item At last, OptiFuzz reports the results of the fuzzing in the form of a PDF report.
\end{itemize}
The OptiFuzz pipeline is illustrated in Figure \ref{fig:optifuzz-pipeline}. 
Each of the steps in the pipeline is described in detail in the following sections.
\begin{figure}[H]
  \centering
  \tikzstyle{box} = [rectangle, minimum width=3cm, minimum height=1cm, text centered, draw=black]
\tikzstyle{arrow} = [thick,->,>=stealth]

\begin{tikzpicture}
  \node (Code Generation) [box] {Code Generation};
  \node (Assembly Inspection) [box, right of=Code Generation, xshift=3cm] {Assembly Inspection};
  \node (Fuzzing) [box, right of=Assembly Inspection, xshift=3cm] {Fuzzing};
  \node (Statistical Analysis) [box, right of=Fuzzing, xshift=3cm] {Statistical Analysis};

  \draw [arrow] (Code Generation) -- (Assembly Inspection);
  \draw [arrow] (Assembly Inspection) -- (Fuzzing);
  \draw [arrow] (Fuzzing) -- (Statistical Analysis);
\end{tikzpicture}
  \caption{The OptiFuzz pipeline.}
  \label{fig:optifuzz-pipeline}
\end{figure}

\subsection{Code Generation}
The first step in the OptiFuzz pipeline is code generation. 
The code generation module is written in OCaml and works by generating abstract syntax trees according to the grammar in Figure \ref{fig:grammar}.

\begin{figure}[H]
  \centering
  \begin{align*}
    e \in Expr ::= & x \mid y &\text{(input variables)}\\
    & n \in \{0, 1\}^{64} &\text{(64-bit integer literals)}\\
    & -e \mid e + e \mid e - e \mid e \times e &\text{(arithmetic operators)}\\
    & \texttt{true} \mid \texttt{false} &\text{(boolean literals)}\\
    & !e &\text{(logical operators)}\\
    & e < e \mid e \leq e \mid e > e \mid e \geq e \mid e = e \mid e \neq e &\text{(comparison operators)}\\
    & e \texttt{\&} e \mid e \texttt{|} e \mid \text{\~{}} e \mid e \text{\^{}} e \mid e \ll e \mid e \gg e &\text{(bitwise operators)}
  \end{align*}
  \caption{The grammar that defines the ASTs generated by the code generation module in the OptiFuzz pipeline.}
  \label{fig:grammar}
\end{figure}

The grammar defines all programs with 2 variables and non-branching arithmetic, logical and comparison operations in C \cite{c-standard}, excluding division (\texttt{/}) and modulus (\texttt{\%}).
The reason for excluding division and modulus is that they may cause division-by-zero errors.
Generated programs are forced to include both input variables, $x$ and $y$.
$x$ and $y$ are the inputs to the program, and we require both to be present since programs with 0 or 1 input variables are trivially constant-time.
Furthermore, Booleans are defined as the integer literals 0 and 1 just like in the C standard library \texttt{stdbool} \cite{c-standard}.
An example of a program generated by the code generation module is shown in Figure \ref{fig:code-gen-example}.

\begin{figure}[H]
  \begin{lstlisting}[style=defstyle,language=C]
#define false 0
#define true 1
int program(int x, int y) { return !(y * (43 * (x != true))); } \end{lstlisting}
  \caption{Example of a program generated by the code generation module.}
  \label{fig:code-gen-example}
\end{figure}

The code generation module works by generating a random distribution that selects different symbols in the grammar with a certain probability.
This ensures that not all generated programs will be uniformly random.
For example, if a generated distribution heavily favors the left-shift operator, then the generated programs will contain a lot of left-shift operations.
This is useful for detecting whether certain types of programs are more likely to be constant-time than others.

The grammar in Figure \ref{fig:grammar} generates programs that are not typical 

\todo{Argue why not a more realistic grammar was chosen}
\todo{Find examples of weird real life programs}
\subsubsection{Limitations}

\subsection{Assembly Inspection}
\label{sec:inspection}

The next step in the OptiFuzz pipeline is assembly inspection.
The assembly inspection module is written in Python and works by inspecting the assembly generated by the compiler across different optimization flags.
The compiled assembly is inspected for conditional branching instructions and flagged if that is the case.
The assembly inspection module is only able to analyze x86-64 assembly.

In x86-64 assembly, \texttt{Jcc} (note, this does not include \texttt{JMP}), \texttt{LOOP} and \texttt{LOOPcc} are the only conditional branching instructions \citep{intel-reference}.
This means that the assembly inspection module only needs to look for these instructions.
Both \texttt{Jcc} and \texttt{LOOPcc} refer to families of instructions where \texttt{cc} is a condition code.
For example, \texttt{JE} (jump if equal) is in the \texttt{Jcc} instruction family. 
We did not include conditional move instructions in the analysis since they are constant-time \citep{cmov-is-constant-time}.

\subsubsection{Limitations}
An obvious limitation of this approach is that it is tailored for x86-64 assembly.
Hence our analysis will not work for other architectures.
Furthermore, the programs that are flagged by the assembly inspection module are not necessarily vulnerable to timing attacks.
Hence, the assembly inspection module overapproximates the set of programs that are vulnerable to timing attacks.
For example, the program in Figure \ref{fig:assembly-inspection-example} is flagged by the assembly inspection module, but it is constant-time since both branches take the same amount of time to execute.

\begin{figure}[]
  \centering
  \begin{lstlisting}[style=defstyle, language={[x86masm]Assembler}, basicstyle=\footnotesize\ttfamily,breaklines=true, xleftmargin=4cm, xrightmargin=4cm]
...
  cmpl    $1, -4(%rbp) ; compare TOS to 1
  je      .L2          ; jump to .L2 if equal
  movl    $43, %eax    ; move 43 into %eax
  jmp     .L3
.L2:
  movl    $0, %eax     ; move 0 into %eax
.L3:
  ret\end{lstlisting}
  \caption{Example of a program that is flagged by the assembly inspection module, but is constant-time.}
  \label{fig:assembly-inspection-example}
\end{figure}
\subsection{Fuzzing}
\todo{Refer to the fact I just wrote that the assembly inspection overapproximates}

\subsubsection{Limitations}
\todo{Underapproximates since fuzzing might not be perfect/noise}
\subsection{Statistical Analysis}
\label{sec:statistical-analysis}
\todo{We set t threshold to 10 like in dudect}
\todo{Analysis based on: https://eprint.iacr.org/2015/536.pdf}
\todo{Remember to say that we only visualize top 95\% percentile since we have some noise. However, importantly we always analyze the whole distribution.}

\section{Experimental Results}
\todo{Note: \url{https://ieeexplore.ieee.org/stamp/stamp.jsp?tp=&arnumber=8077809} says clang had issues with not compiling to cmov back in the day (2017)}
All experiments were carried out on an Intel Core i7-9750H running Manjaro Linux 22.1.2 with kernel version Linux 5.15.112-1.

\subsection{Genereal Optimizations}
\todo{O0, ..., Os}

\subsection{Specific Optimizations}
\todo{Single out specific optimizations}

\subsection{Vulnerable operations}
\todo{Try to identify what operations are vulnerable}


\section{Evaluation}
\todo{Evaluation of OptiFuzz and analysis. Is it accurate?}

\section{Solving the Problem}
From the previous sections, it is clear that modern compilers introduce timing vulnerabilities in constant-time programs.
This is devastating for security since it means that developers cannot reliably write constant-time code.
A solution is required.
In this section, we discuss how the problem can be solved and what the implications different solutions have on security and efficiency.

As shown in Section \ref{sec:gcc-vs-clang}, \texttt{clang} is much more conservative when it comes to inserting conditional branches in constant-time programs.
Instead of inserting variable-time conditional branches, \texttt{clang} uses constant-time conditional moves to implement branching control flow.
From this observation, it might be tempting to conclude that always substituting conditional branches with conditional moves is the solution to the problem.
However, there is more to the story.

\texttt{cmov} instructions severely restrict the out-of-order engine on modern CPUs since \texttt{cmov} instructions increase the data dependency between instructions significantly \citep{intel-optimization-reference}. 
Furthermore, branch predictions are not used for \texttt{cmov} instructions, which can lead to significant performance degradation.
Hence, we have to address this trade-off between security and efficiency.

\subsection{Forcing \texttt{cmov} Instructions When Necessary}
This trade-off between security and efficiency has led multiple researchers to propose solutions that only force \texttt{cmov} instructions to be used on critical data \citep{what-you-c,llvm-issues-blog-post}.
In both studies, the authors propose to add the possibility to specify forced \texttt{cmov} instructions in the source code and restrict the backend of the compiler to not being able to optimize the \texttt{cmov} away for those specific instructions.
This solution has the advantage that it does not affect the performance of non-critical parts of the program.
In fact, \citeauthor{what-you-c} show that their solution only introduces a performance penalty of 1\% and sometimes even increases performance.

However, this solution also has its disadvantages.
First, the solution relies heavily on the developer.
The developer has to know which parts of the program are critical, and also be able to use the language extension correctly.
In turn, it means that security relies on human factors, which are error-prone.
Second, the solution relies on the compiler being configurable to avoid forced \texttt{cmov} instructions being optimized away.
As we discussed in Section \ref{sec:gcc-vs-clang} and \ref{sec:general-optimizations}, \texttt{clang} seems to be quite configurable while \texttt{gcc} does not.
Hence, the solution might not be easily applicable to all compilers.

Another solution is to implement vulnerable parts of the program using a domain-specific language that is designed to be constant-time.
\citeauthor{fact} proposed a domain-specific language, called FaCT, for implementing constant-time cryptographic algorithms \citep{fact}.
FaCT is designed to be interoperable with C, and the compiler ensures constant-time code by leveraging the \texttt{ct-verif} tool that is based on verification of the LLVM IR \cite{verifying-constant-time-llvm}.
\texttt{ct-verif} operates on a formalization of constant-time code and verifies based on a theoretically sound and complete methodology.
However, \citeauthor{verifying-constant-time-llvm} note that the verification process of \texttt{ct-verif} is not guaranteed to identify all timing vulnerabilities due to translation between the LLVM IR and machine code, and due to possible incompleteness of their formalization of constant-time code. 
Some of these issues were experimentally verified.
However, generally \texttt{ct-verif} captures significantly more timing vulnerabilities than other verification alternatives \citep{verifying-constant-time-llvm}.
Additionally, \texttt{ct-verif} is efficient at verifying constant-time code, often being much faster than the compilation step, but occasionally being a few times slower.

\subsection{Bullet-proofing the Compiler}
A more radical approach is to force the compiler to always produce constant-time code.
\citeauthor{verified-constant-time-c-comiler} provides a formally verified constant-time C compiler, built on top of the CompCert verified C compiler \citep{verified-constant-time-c-comiler}. 
The constant-time CompCert compiler guarantees that constant-time source code is compiled into constant-time machine code.
This is achieved by instrumenting the operational semantics of CompCert's IR to be able to capture cryptographic constant-time and proving that cryptographic constant-time is preserved during compilation. 
This has immense advantages for security.

However, the security advantages come at the cost of efficiency. 
The constant-time CompCert compiler is significantly slower than \texttt{gcc}.
Experimental results show an efficiency penalty of well over 100\% in some cases \citep{verified-constant-time-c-comiler}.

\subsection{Testing Solutions}

\todo{Black-box solutions. Argue that it is too vague, but perhaps OptiFuzz can be used to minimize issues from compilers.}

\section{Conclusion and Future Work}
\todo{Future Work: More complicated programs}

\pagebreak

\bibliographystyle{abbrvnat}
\bibliography{litterature}

\pagebreak
\begin{appendices}

\section{Select General Optimizations Results}
\label{appendix:general-optimizations-results}
%\documentclass[10pt]{article}
\usepackage[left=0.8in,right=0.8in,bottom=0.5in,top=0.5in]{geometry}
\geometry{a4paper}
\usepackage[parfill]{parskip}    % Activate to begin paragraphs with an empty line rather than an indent
\usepackage{graphicx}
\usepackage{xcolor}
\usepackage{hyperref}
\usepackage{titling}
\usepackage[small,compact]{titlesec}
\usepackage[toc,page]{appendix}

\usepackage{listings}
\usepackage{pgfplots}
\usepackage{adjustbox}
\usepackage{float}
\usepackage{subcaption}
\usepackage{pgffor}
\usepackage{expl3}
\usepackage{xparse}
\usepackage{xfp}
\usetikzlibrary{fit}
\usetikzlibrary{positioning}


\usepackage{amsmath,amssymb}
\DeclareGraphicsRule{.tif}{png}{.png}{`convert #1 `dirname #1`/`basename #1 .tif`.png}



\newcommand\courseName{Language-Based Security}
\newcommand\courseNameAbbrv{LBS}
\newcommand\courseYear{2023}
\newcommand\courseAndYear{\courseNameAbbrv-\courseYear}
\newcommand\reportKind{Exam Report}

\newcommand\groupNumber[1]{
  \makeatletter
  \def\@courseGroupNumber{#1}
  \makeatother
}

\pretitle{\begin{flushright} \bfseries  \large \courseAndYear: \reportKind \end{flushright}   \begin{flushleft}
\bfseries \LARGE}
\posttitle{\par\end{flushleft}\vskip 0.5em}

\preauthor{
\begin{flushleft}
\large \lineskip 0.5em%
\bfseries
\makeatletter Group \@courseGroupNumber \\ \makeatother
\begin{tabular}[t]{@{}l}}
\postauthor{\end{tabular}\par\end{flushleft}}

\predate{\begin{flushleft}\bfseries \large}
\postdate{\par\end{flushleft}}

\usepackage{listings}
\usepackage{pgfplots}
\usepackage{adjustbox}
\usepackage{float}
\usepackage{subcaption}
\usepackage{pgffor}
\usepackage{expl3}
\usepackage{xparse}
\usepackage{xfp}

\newcommand\todo[1]{\textcolor{red}{TODO: #1}}

\pgfplotsset{compat=1.12}

\definecolor{mGreen}{rgb}{0,0.6,0}
\definecolor{mGray}{rgb}{0.5,0.5,0.5}
\definecolor{mPurple}{rgb}{0.58,0,0.82}
\definecolor{backgroundColour}{rgb}{0.95,0.95,0.92}

\definecolor{firstCol}{HTML}{332288}
\definecolor{secondCol}{HTML}{117733}
\definecolor{thirdCol}{HTML}{44AA99}
\definecolor{fourthCol}{HTML}{88CCEE}
\definecolor{fifthCol}{HTML}{DDCC77}
\definecolor{sixthCol}{HTML}{DD2255}

\lstdefinestyle{defstyle}{
    backgroundcolor=\color{backgroundColour},   
    commentstyle=\color{mGreen},
    keywordstyle=\color{magenta},
    numberstyle=\tiny\color{mGray},
    stringstyle=\color{mPurple},
    basicstyle=\footnotesize,
    breakatwhitespace=false,         
    breaklines=true,                 
    captionpos=b,                    
    keepspaces=true,                 
    numbers=left,                    
    numbersep=5pt,                  
    showspaces=false,                
    showstringspaces=false,
    showtabs=false,                  
    tabsize=2
}

\newsavebox{\mybox}

\ExplSyntaxOn
\cs_new:Npn \__afp_ismember_loop:Nnw #1#2#3,
  {
    \quark_if_recursion_tail_stop_do:nn {#3}
      { \prg_return_false: }
    #1 {#2} {#3}
      { \use_i_delimit_by_q_recursion_stop:nw { \prg_return_true: } }
      { \__afp_ismember_loop:Nnw #1 {#2} }
  }
\prg_new_conditional:Npnn \afp_int_ismember:nn #1#2 { p, T, F, TF }
  {
    \__afp_ismember_loop:Nnw \__afp_int_isequal:nnTF {#1} #2 ,
    \q_recursion_tail , \q_recursion_stop
  }
\prg_new_conditional:Npnn \__afp_int_isequal:nn #1#2 { p, T, F, TF }
  {
    \int_compare:nNnTF {#1} = {#2}
      { \prg_return_true: }
      { \prg_return_false: }
  }
\NewExpandableDocumentCommand { \IncludeExperimentResults } { m m }
  {
    \foreach \x in {1,...,#1} {
        \noindent\bool_if:NF { \afp_int_ismember_p:nn {\x} {#2} } {\input{generated_latex/prog\x.tex}}
    }
  }
\ExplSyntaxOff
\usepackage[authoryear]{natbib}

\title{
  Optimizing Away Security in C
}

\groupNumber{10}
\author{Anders B. Clausen \and Johan T. Degn \and Jonathan S. Eilath}

\begin{document}
\maketitle
\thispagestyle{empty}

\todo{Add short guide to read data plots}
\todo{We can probably add "what compile flags are dangerous" to contributions}
\todo{Make figures smaller and better, and remove [H]}

\section*{Abstract}
\todo{After the paper is done}
\todo{Write something like: the status quo is the worst of both worlds}

\section{Introduction}
\subsection{Related Work}
\cite{what-you-c}
\subsection{Contributions}
\subsection{Overview}

\section{Preliminaries}
\label{sec:preliminaries}
\subsection{Timing Attacks}
Timing attacks are a class of side-channel attacks that exploit the fact that the execution time of a program can depend on the input.
The history of timing attacks goes back several decades where Kocher showed multiple successful timing attacks on well-known cryptographic algorithms such as Diffie-Hellman and RSA \citep{1996-timing-attacks}.
An example of vulnerable code is shown in Figure \ref{fig:timing-attack-example}.
\begin{figure}[H]
  \begin{lstlisting}[style=defstyle,language=C, xleftmargin=6.8cm, xrightmargin=6.8cm]
int foo(int x) {
  if (x < 100) {
    x *= 2;
    x += 7;
  }
  return x;
} \end{lstlisting} 
  \caption{Example of a program vulnerable to a timing attack. 
  Only by analyzing the execution time of the machine code, an attacker can infer whether the input is less than 100 or not.}
  \label{fig:timing-attack-example}
\end{figure}

\subsection{Optimizing Compilers}
Cryptographers will avoid code like the example in Figure \ref{fig:timing-attack-example} and write constant-time code instead.
Code is constant-time if the execution time of the program is independent of the input.
However, the compiler may introduce timing vulnerabilities through optimizations by adding variable-time branches to the machine code.
The issue arises in the analysis and transformation phases of the compiler as illustrated in Figure \ref{fig:optimizing-compiler-pipeline}.

\begin{figure}[H]
  \centering
  \tikzstyle{box} = [rectangle, minimum width=2.8cm, minimum height=0.7cm, text centered, draw=black]
\tikzstyle{arrow} = [thick,->,>=stealth]

\begin{tikzpicture}
  \node (Source Code) [box] {\small Source Code};
  \node (Lexing) [box, fill=lightgray, right of=Source Code, xshift=2.5cm] {\small Lexing};
  \node (Parsing) [box, fill=lightgray, right of=Lexing, xshift=2.5cm] {\small Parsing};
  \node (AST) [box, right of = Parsing, xshift=2.5cm] {\small AST};
  \node (Semantic Analysis) [box, fill=lightgray, right of=AST, xshift=2.5cm] {\small Semantic Analysis};
  \node (Intermediate Representation) [box, below of=Semantic Analysis, yshift=-1.2cm, xshift=0cm] {\small IR};
  \node (Analysis) [box, fill=lightgray, left of=Intermediate Representation, xshift=-2.5cm] {\small Analysis};
  \node (Transformation) [box, fill=lightgray, left of=Analysis, xshift=-2.5cm] {\small Transformation};
  \node (Code Generation) [box, fill=lightgray, left of=Transformation, xshift=-2.5cm] {\small Code Generation};
  \node (Executable) [box, left of=Code Generation, xshift=-2.5cm] {\small Executable};

  \draw [arrow] (Source Code) -- (Lexing);
  \draw [arrow] (Lexing) -- (Parsing);
  \draw [arrow] (Parsing) -- node [fill=white, text=darkgray, yshift=0.7cm] {\small Syntactically Sound} (AST);
  \draw [arrow] (AST) -- (Semantic Analysis);
  \draw [arrow] (Semantic Analysis) -- node [fill=white, text=darkgray] {\small Semantically Sound} (Intermediate Representation);
  \draw [arrow] (Intermediate Representation) -- (Analysis);
  \draw [arrow] (Analysis) -- (Transformation);
  \draw [arrow, dashed] (Transformation) to [bend left=25] (Intermediate Representation);
  \draw [arrow] (Transformation) -- node [fill=white, text=red, yshift=0.7cm] {\small Optimized} (Code Generation);
  \draw [arrow] (Code Generation) -- (Executable);
\end{tikzpicture}
  \caption{The pipeline of an optimizing compiler. After the transformation phase, the IR is optimized and the compiler may have introduced timing vulnerabilities.}
  \label{fig:optimizing-compiler-pipeline}
\end{figure}

Many different kinds of optimization techniques are carried out by optimizing compilers, some of which can introduce timing vulnerabilities.
Some common optimization techniques like common subexpression elimination and strength reduction have been shown to introduce timing vulnerabilities \citep{optimizations-linked-to-timing-attacks}.
To illustrate this point, we look at how common subexpression elimination can introduce timing vulnerabilities.

\subsubsection{Timing Vulnerabilities Through Common Subexpression Elimination}
\label{sec:cse}
Common Subexpression Elimination is an optimization technique that extracts subexpressions that are common across multiple expressions and replaces them with a single variable.
This optimization technique can introduce timing vulnerabilities since it can decrease the number of instructions executed for a specific branch of the code as illustrated in Figure \ref{fig:common-subexpression-elimination}.
Here the common subexpression \texttt{2 * 3 + 5} is extracted and assigned to the variable \texttt{common}, making the \texttt{else}-branch execute faster than the \texttt{then}-branch.

\begin{figure}[H]
  \centering
     \begin{subfigure}[b]{0.3\textwidth}
        \begin{lstlisting}[style=defstyle, language=C]
int foo(int x, int *arr) {
  if (x == SECRET) {
    x = arr[0] * 3 + 5;
    x += arr[1] * 3 + 5;
    x += arr[2] * 3 + 5;
  } else {
    x = 2 * 3 + 5;
    x += 2 * 3 + 5;
    x += 2 * 3 + 5;
  }
  return x;
} \end{lstlisting} 
         \caption{Original code.}
    \end{subfigure}
    \hspace{1cm}
    \begin{subfigure}[b]{0.3\textwidth}
      \begin{lstlisting}[style=defstyle, language=C]
int foo(int x, int *arr) {
  if (x == SECRET) {
    x = arr[0] * 3 + 5;
    x += arr[1] * 3 + 5;
    x += arr[2] * 3 + 5;
  } else {
    // optimized
    int common = 2 * 3 + 5;
    x = 3 * common;
  }
  return x;
} \end{lstlisting} 
       \caption{Optimized code.}
  \end{subfigure}
  \caption{An example of how the common subexpression elimination can introduce timing vulnerabilities in code.}
  \label{fig:common-subexpression-elimination}
\end{figure}

\section{OptiFuzz}
% I propose "generate, fuzz and analyze" as this conveys our pipeline
We created a tool, OptiFuzz, that can be used to generate, analyze and fuzz random C programs.
The source code for OptiFuzz is available on GitHub\footnote{\url{https://github.com/anbclausen/optifuzz}.}.
The goal of OptiFuzz is to quantify the issue of timing attacks introduced by C compilers with different optimization flags enabled.
The tool works as follows:
\begin{itemize}
  \item OptiFuzz generates random C programs consisting of non-branching arithmetic, logical and comparison operations.
  \item OptiFuzz then compiles the generated C programs with different specified optimization flags enabled and inspects the generated assembly for conditional branching instructions introduced by the compiler.
        If branching is found, the program is flagged.
  \item OptiFuzz then fuzzes the flagged programs with various random inputs to test whether the branching instructions can be exploited to leak information about the input.
  \item At last, OptiFuzz reports the results of the fuzzing in the form of a PDF report.
\end{itemize}
The OptiFuzz pipeline is illustrated in Figure \ref{fig:optifuzz-pipeline}. 
Each of the steps in the pipeline is described in detail in the following sections.
\begin{figure}[H]
  \centering
  \tikzstyle{box} = [rectangle, minimum width=3cm, minimum height=1cm, text centered, draw=black]
\tikzstyle{arrow} = [thick,->,>=stealth]

\begin{tikzpicture}
  \node (Code Generation) [box] {Code Generation};
  \node (Assembly Inspection) [box, right of=Code Generation, xshift=3cm] {Assembly Inspection};
  \node (Fuzzing) [box, right of=Assembly Inspection, xshift=3cm] {Fuzzing};
  \node (Statistical Analysis) [box, right of=Fuzzing, xshift=3cm] {Statistical Analysis};

  \draw [arrow] (Code Generation) -- (Assembly Inspection);
  \draw [arrow] (Assembly Inspection) -- (Fuzzing);
  \draw [arrow] (Fuzzing) -- (Statistical Analysis);
\end{tikzpicture}
  \caption{The OptiFuzz pipeline.}
  \label{fig:optifuzz-pipeline}
\end{figure}

\subsection{Code Generation}
The first step in the OptiFuzz pipeline is code generation. 
The code generation module is written in OCaml and works by generating abstract syntax trees according to the grammar in Figure \ref{fig:grammar}.

\begin{figure}[H]
  \centering
  \begin{align*}
    e \in Expr ::= & x \mid y &\text{(input variables)}\\
    & n \in \{0, 1\}^{64} &\text{(64-bit integer literals)}\\
    & -e \mid e + e \mid e - e \mid e \times e &\text{(arithmetic operators)}\\
    & \texttt{true} \mid \texttt{false} &\text{(boolean literals)}\\
    & !e &\text{(logical operators)}\\
    & e < e \mid e \leq e \mid e > e \mid e \geq e \mid e = e \mid e \neq e &\text{(comparison operators)}\\
    & e \texttt{\&} e \mid e \texttt{|} e \mid \text{\~{}} e \mid e \text{\^{}} e \mid e \ll e \mid e \gg e &\text{(bitwise operators)}
  \end{align*}
  \caption{The grammar that defines the ASTs generated by the code generation module in the OptiFuzz pipeline.}
  \label{fig:grammar}
\end{figure}

The grammar defines all programs with 2 variables and non-branching arithmetic, logical and comparison operations in C \cite{c-standard}, excluding division (\texttt{/}) and modulus (\texttt{\%}).
The reason for excluding division and modulus is that they may cause division-by-zero errors.
Generated programs are forced to include both input variables, $x$ and $y$.
$x$ and $y$ are the inputs to the program, and we require both to be present since programs with 0 or 1 input variables are trivially constant-time.
Furthermore, Booleans are defined as the integer literals 0 and 1 just like in the C standard library \texttt{stdbool} \cite{c-standard}.
An example of a program generated by the code generation module is shown in Figure \ref{fig:code-gen-example}.

\begin{figure}[H]
  \begin{lstlisting}[style=defstyle,language=C]
#define false 0
#define true 1
int program(int x, int y) { return !(y * (43 * (x != true))); } \end{lstlisting}
  \caption{Example of a program generated by the code generation module.}
  \label{fig:code-gen-example}
\end{figure}

The code generation module works by generating a random distribution that selects different symbols in the grammar with a certain probability.
This ensures that not all generated programs will be uniformly random.
For example, if a generated distribution heavily favors the left-shift operator, then the generated programs will contain a lot of left-shift operations.
This is useful for detecting whether certain types of programs are more likely to be constant-time than others.

The grammar in Figure \ref{fig:grammar} generates programs that are not typical 

\todo{Argue why not a more realistic grammar was chosen}
\todo{Find examples of weird real life programs}
\subsubsection{Limitations}

\subsection{Assembly Inspection}
\label{sec:inspection}

The next step in the OptiFuzz pipeline is assembly inspection.
The assembly inspection module is written in Python and works by inspecting the assembly generated by the compiler across different optimization flags.
The compiled assembly is inspected for conditional branching instructions and flagged if that is the case.
The assembly inspection module is only able to analyze x86-64 assembly.

In x86-64 assembly, \texttt{Jcc} (note, this does not include \texttt{JMP}), \texttt{LOOP} and \texttt{LOOPcc} are the only conditional branching instructions \citep{intel-reference}.
This means that the assembly inspection module only needs to look for these instructions.
Both \texttt{Jcc} and \texttt{LOOPcc} refer to families of instructions where \texttt{cc} is a condition code.
For example, \texttt{JE} (jump if equal) is in the \texttt{Jcc} instruction family. 
We did not include conditional move instructions in the analysis since they are constant-time \citep{cmov-is-constant-time}.

\subsubsection{Limitations}
An obvious limitation of this approach is that it is tailored for x86-64 assembly.
Hence our analysis will not work for other architectures.
Furthermore, the programs that are flagged by the assembly inspection module are not necessarily vulnerable to timing attacks.
Hence, the assembly inspection module overapproximates the set of programs that are vulnerable to timing attacks.
For example, the program in Figure \ref{fig:assembly-inspection-example} is flagged by the assembly inspection module, but it is constant-time since both branches take the same amount of time to execute.

\begin{figure}[]
  \centering
  \begin{lstlisting}[style=defstyle, language={[x86masm]Assembler}, basicstyle=\footnotesize\ttfamily,breaklines=true, xleftmargin=4cm, xrightmargin=4cm]
...
  cmpl    $1, -4(%rbp) ; compare TOS to 1
  je      .L2          ; jump to .L2 if equal
  movl    $43, %eax    ; move 43 into %eax
  jmp     .L3
.L2:
  movl    $0, %eax     ; move 0 into %eax
.L3:
  ret\end{lstlisting}
  \caption{Example of a program that is flagged by the assembly inspection module, but is constant-time.}
  \label{fig:assembly-inspection-example}
\end{figure}
\subsection{Fuzzing}
\todo{Refer to the fact I just wrote that the assembly inspection overapproximates}

\subsubsection{Limitations}
\todo{Underapproximates since fuzzing might not be perfect/noise}
\subsection{Statistical Analysis}
\label{sec:statistical-analysis}
\todo{We set t threshold to 10 like in dudect}
\todo{Analysis based on: https://eprint.iacr.org/2015/536.pdf}
\todo{Remember to say that we only visualize top 95\% percentile since we have some noise. However, importantly we always analyze the whole distribution.}

\section{Experimental Results}
\todo{Note: \url{https://ieeexplore.ieee.org/stamp/stamp.jsp?tp=&arnumber=8077809} says clang had issues with not compiling to cmov back in the day (2017)}
All experiments were carried out on an Intel Core i7-9750H running Manjaro Linux 22.1.2 with kernel version Linux 5.15.112-1.

\subsection{Genereal Optimizations}
\todo{O0, ..., Os}

\subsection{Specific Optimizations}
\todo{Single out specific optimizations}

\subsection{Vulnerable operations}
\todo{Try to identify what operations are vulnerable}


\section{Evaluation}
\todo{Evaluation of OptiFuzz and analysis. Is it accurate?}

\section{Solving the Problem}
From the previous sections, it is clear that modern compilers introduce timing vulnerabilities in constant-time programs.
This is devastating for security since it means that developers cannot reliably write constant-time code.
A solution is required.
In this section, we discuss how the problem can be solved and what the implications different solutions have on security and efficiency.

As shown in Section \ref{sec:gcc-vs-clang}, \texttt{clang} is much more conservative when it comes to inserting conditional branches in constant-time programs.
Instead of inserting variable-time conditional branches, \texttt{clang} uses constant-time conditional moves to implement branching control flow.
From this observation, it might be tempting to conclude that always substituting conditional branches with conditional moves is the solution to the problem.
However, there is more to the story.

\texttt{cmov} instructions severely restrict the out-of-order engine on modern CPUs since \texttt{cmov} instructions increase the data dependency between instructions significantly \citep{intel-optimization-reference}. 
Furthermore, branch predictions are not used for \texttt{cmov} instructions, which can lead to significant performance degradation.
Hence, we have to address this trade-off between security and efficiency.

\subsection{Forcing \texttt{cmov} Instructions When Necessary}
This trade-off between security and efficiency has led multiple researchers to propose solutions that only force \texttt{cmov} instructions to be used on critical data \citep{what-you-c,llvm-issues-blog-post}.
In both studies, the authors propose to add the possibility to specify forced \texttt{cmov} instructions in the source code and restrict the backend of the compiler to not being able to optimize the \texttt{cmov} away for those specific instructions.
This solution has the advantage that it does not affect the performance of non-critical parts of the program.
In fact, \citeauthor{what-you-c} show that their solution only introduces a performance penalty of 1\% and sometimes even increases performance.

However, this solution also has its disadvantages.
First, the solution relies heavily on the developer.
The developer has to know which parts of the program are critical, and also be able to use the language extension correctly.
In turn, it means that security relies on human factors, which are error-prone.
Second, the solution relies on the compiler being configurable to avoid forced \texttt{cmov} instructions being optimized away.
As we discussed in Section \ref{sec:gcc-vs-clang} and \ref{sec:general-optimizations}, \texttt{clang} seems to be quite configurable while \texttt{gcc} does not.
Hence, the solution might not be easily applicable to all compilers.

Another solution is to implement vulnerable parts of the program using a domain-specific language that is designed to be constant-time.
\citeauthor{fact} proposed a domain-specific language, called FaCT, for implementing constant-time cryptographic algorithms \citep{fact}.
FaCT is designed to be interoperable with C, and the compiler ensures constant-time code by leveraging the \texttt{ct-verif} tool that is based on verification of the LLVM IR \cite{verifying-constant-time-llvm}.
\texttt{ct-verif} operates on a formalization of constant-time code and verifies based on a theoretically sound and complete methodology.
However, \citeauthor{verifying-constant-time-llvm} note that the verification process of \texttt{ct-verif} is not guaranteed to identify all timing vulnerabilities due to translation between the LLVM IR and machine code, and due to possible incompleteness of their formalization of constant-time code. 
Some of these issues were experimentally verified.
However, generally \texttt{ct-verif} captures significantly more timing vulnerabilities than other verification alternatives \citep{verifying-constant-time-llvm}.
Additionally, \texttt{ct-verif} is efficient at verifying constant-time code, often being much faster than the compilation step, but occasionally being a few times slower.

\subsection{Bullet-proofing the Compiler}
A more radical approach is to force the compiler to always produce constant-time code.
\citeauthor{verified-constant-time-c-comiler} provides a formally verified constant-time C compiler, built on top of the CompCert verified C compiler \citep{verified-constant-time-c-comiler}. 
The constant-time CompCert compiler guarantees that constant-time source code is compiled into constant-time machine code.
This is achieved by instrumenting the operational semantics of CompCert's IR to be able to capture cryptographic constant-time and proving that cryptographic constant-time is preserved during compilation. 
This has immense advantages for security.

However, the security advantages come at the cost of efficiency. 
The constant-time CompCert compiler is significantly slower than \texttt{gcc}.
Experimental results show an efficiency penalty of well over 100\% in some cases \citep{verified-constant-time-c-comiler}.

\subsection{Testing Solutions}

\todo{Black-box solutions. Argue that it is too vague, but perhaps OptiFuzz can be used to minimize issues from compilers.}

\section{Conclusion and Future Work}
\todo{Future Work: More complicated programs}

\pagebreak

\bibliographystyle{abbrvnat}
\bibliography{litterature}

\pagebreak
\begin{appendices}

\section{Select General Optimizations Results}
\label{appendix:general-optimizations-results}
%\documentclass[10pt]{article}
\usepackage[left=0.8in,right=0.8in,bottom=0.5in,top=0.5in]{geometry}
\geometry{a4paper}
\usepackage[parfill]{parskip}    % Activate to begin paragraphs with an empty line rather than an indent
\usepackage{graphicx}
\usepackage{xcolor}
\usepackage{hyperref}
\usepackage{titling}
\usepackage[small,compact]{titlesec}
\usepackage[toc,page]{appendix}

\usepackage{listings}
\usepackage{pgfplots}
\usepackage{adjustbox}
\usepackage{float}
\usepackage{subcaption}
\usepackage{pgffor}
\usepackage{expl3}
\usepackage{xparse}
\usepackage{xfp}
\usetikzlibrary{fit}
\usetikzlibrary{positioning}


\usepackage{amsmath,amssymb}
\DeclareGraphicsRule{.tif}{png}{.png}{`convert #1 `dirname #1`/`basename #1 .tif`.png}



\newcommand\courseName{Language-Based Security}
\newcommand\courseNameAbbrv{LBS}
\newcommand\courseYear{2023}
\newcommand\courseAndYear{\courseNameAbbrv-\courseYear}
\newcommand\reportKind{Exam Report}

\newcommand\groupNumber[1]{
  \makeatletter
  \def\@courseGroupNumber{#1}
  \makeatother
}

\pretitle{\begin{flushright} \bfseries  \large \courseAndYear: \reportKind \end{flushright}   \begin{flushleft}
\bfseries \LARGE}
\posttitle{\par\end{flushleft}\vskip 0.5em}

\preauthor{
\begin{flushleft}
\large \lineskip 0.5em%
\bfseries
\makeatletter Group \@courseGroupNumber \\ \makeatother
\begin{tabular}[t]{@{}l}}
\postauthor{\end{tabular}\par\end{flushleft}}

\predate{\begin{flushleft}\bfseries \large}
\postdate{\par\end{flushleft}}

\usepackage{listings}
\usepackage{pgfplots}
\usepackage{adjustbox}
\usepackage{float}
\usepackage{subcaption}
\usepackage{pgffor}
\usepackage{expl3}
\usepackage{xparse}
\usepackage{xfp}

\newcommand\todo[1]{\textcolor{red}{TODO: #1}}

\pgfplotsset{compat=1.12}

\definecolor{mGreen}{rgb}{0,0.6,0}
\definecolor{mGray}{rgb}{0.5,0.5,0.5}
\definecolor{mPurple}{rgb}{0.58,0,0.82}
\definecolor{backgroundColour}{rgb}{0.95,0.95,0.92}

\definecolor{firstCol}{HTML}{332288}
\definecolor{secondCol}{HTML}{117733}
\definecolor{thirdCol}{HTML}{44AA99}
\definecolor{fourthCol}{HTML}{88CCEE}
\definecolor{fifthCol}{HTML}{DDCC77}
\definecolor{sixthCol}{HTML}{DD2255}

\lstdefinestyle{defstyle}{
    backgroundcolor=\color{backgroundColour},   
    commentstyle=\color{mGreen},
    keywordstyle=\color{magenta},
    numberstyle=\tiny\color{mGray},
    stringstyle=\color{mPurple},
    basicstyle=\footnotesize,
    breakatwhitespace=false,         
    breaklines=true,                 
    captionpos=b,                    
    keepspaces=true,                 
    numbers=left,                    
    numbersep=5pt,                  
    showspaces=false,                
    showstringspaces=false,
    showtabs=false,                  
    tabsize=2
}

\newsavebox{\mybox}

\ExplSyntaxOn
\cs_new:Npn \__afp_ismember_loop:Nnw #1#2#3,
  {
    \quark_if_recursion_tail_stop_do:nn {#3}
      { \prg_return_false: }
    #1 {#2} {#3}
      { \use_i_delimit_by_q_recursion_stop:nw { \prg_return_true: } }
      { \__afp_ismember_loop:Nnw #1 {#2} }
  }
\prg_new_conditional:Npnn \afp_int_ismember:nn #1#2 { p, T, F, TF }
  {
    \__afp_ismember_loop:Nnw \__afp_int_isequal:nnTF {#1} #2 ,
    \q_recursion_tail , \q_recursion_stop
  }
\prg_new_conditional:Npnn \__afp_int_isequal:nn #1#2 { p, T, F, TF }
  {
    \int_compare:nNnTF {#1} = {#2}
      { \prg_return_true: }
      { \prg_return_false: }
  }
\NewExpandableDocumentCommand { \IncludeExperimentResults } { m m }
  {
    \foreach \x in {1,...,#1} {
        \noindent\bool_if:NF { \afp_int_ismember_p:nn {\x} {#2} } {\input{generated_latex/prog\x.tex}}
    }
  }
\ExplSyntaxOff
\usepackage[authoryear]{natbib}

\title{
  Optimizing Away Security in C
}

\groupNumber{10}
\author{Anders B. Clausen \and Johan T. Degn \and Jonathan S. Eilath}

\begin{document}
\maketitle
\thispagestyle{empty}

\todo{Add short guide to read data plots}
\todo{We can probably add "what compile flags are dangerous" to contributions}
\todo{Make figures smaller and better, and remove [H]}

\section*{Abstract}
\todo{After the paper is done}
\todo{Write something like: the status quo is the worst of both worlds}

\section{Introduction}
\subsection{Related Work}
\cite{what-you-c}
\subsection{Contributions}
\subsection{Overview}

\section{Preliminaries}
\label{sec:preliminaries}
\subsection{Timing Attacks}
Timing attacks are a class of side-channel attacks that exploit the fact that the execution time of a program can depend on the input.
The history of timing attacks goes back several decades where Kocher showed multiple successful timing attacks on well-known cryptographic algorithms such as Diffie-Hellman and RSA \citep{1996-timing-attacks}.
An example of vulnerable code is shown in Figure \ref{fig:timing-attack-example}.
\begin{figure}[H]
  \begin{lstlisting}[style=defstyle,language=C, xleftmargin=6.8cm, xrightmargin=6.8cm]
int foo(int x) {
  if (x < 100) {
    x *= 2;
    x += 7;
  }
  return x;
} \end{lstlisting} 
  \caption{Example of a program vulnerable to a timing attack. 
  Only by analyzing the execution time of the machine code, an attacker can infer whether the input is less than 100 or not.}
  \label{fig:timing-attack-example}
\end{figure}

\subsection{Optimizing Compilers}
Cryptographers will avoid code like the example in Figure \ref{fig:timing-attack-example} and write constant-time code instead.
Code is constant-time if the execution time of the program is independent of the input.
However, the compiler may introduce timing vulnerabilities through optimizations by adding variable-time branches to the machine code.
The issue arises in the analysis and transformation phases of the compiler as illustrated in Figure \ref{fig:optimizing-compiler-pipeline}.

\begin{figure}[H]
  \centering
  \input{tikz/optimizing-compiler-pipeline.tex}
  \caption{The pipeline of an optimizing compiler. After the transformation phase, the IR is optimized and the compiler may have introduced timing vulnerabilities.}
  \label{fig:optimizing-compiler-pipeline}
\end{figure}

Many different kinds of optimization techniques are carried out by optimizing compilers, some of which can introduce timing vulnerabilities.
Some common optimization techniques like common subexpression elimination and strength reduction have been shown to introduce timing vulnerabilities \citep{optimizations-linked-to-timing-attacks}.
To illustrate this point, we look at how common subexpression elimination can introduce timing vulnerabilities.

\subsubsection{Timing Vulnerabilities Through Common Subexpression Elimination}
\label{sec:cse}
Common Subexpression Elimination is an optimization technique that extracts subexpressions that are common across multiple expressions and replaces them with a single variable.
This optimization technique can introduce timing vulnerabilities since it can decrease the number of instructions executed for a specific branch of the code as illustrated in Figure \ref{fig:common-subexpression-elimination}.
Here the common subexpression \texttt{2 * 3 + 5} is extracted and assigned to the variable \texttt{common}, making the \texttt{else}-branch execute faster than the \texttt{then}-branch.

\begin{figure}[H]
  \centering
     \begin{subfigure}[b]{0.3\textwidth}
        \begin{lstlisting}[style=defstyle, language=C]
int foo(int x, int *arr) {
  if (x == SECRET) {
    x = arr[0] * 3 + 5;
    x += arr[1] * 3 + 5;
    x += arr[2] * 3 + 5;
  } else {
    x = 2 * 3 + 5;
    x += 2 * 3 + 5;
    x += 2 * 3 + 5;
  }
  return x;
} \end{lstlisting} 
         \caption{Original code.}
    \end{subfigure}
    \hspace{1cm}
    \begin{subfigure}[b]{0.3\textwidth}
      \begin{lstlisting}[style=defstyle, language=C]
int foo(int x, int *arr) {
  if (x == SECRET) {
    x = arr[0] * 3 + 5;
    x += arr[1] * 3 + 5;
    x += arr[2] * 3 + 5;
  } else {
    // optimized
    int common = 2 * 3 + 5;
    x = 3 * common;
  }
  return x;
} \end{lstlisting} 
       \caption{Optimized code.}
  \end{subfigure}
  \caption{An example of how the common subexpression elimination can introduce timing vulnerabilities in code.}
  \label{fig:common-subexpression-elimination}
\end{figure}

\section{OptiFuzz}
% I propose "generate, fuzz and analyze" as this conveys our pipeline
We created a tool, OptiFuzz, that can be used to generate, analyze and fuzz random C programs.
The source code for OptiFuzz is available on GitHub\footnote{\url{https://github.com/anbclausen/optifuzz}.}.
The goal of OptiFuzz is to quantify the issue of timing attacks introduced by C compilers with different optimization flags enabled.
The tool works as follows:
\begin{itemize}
  \item OptiFuzz generates random C programs consisting of non-branching arithmetic, logical and comparison operations.
  \item OptiFuzz then compiles the generated C programs with different specified optimization flags enabled and inspects the generated assembly for conditional branching instructions introduced by the compiler.
        If branching is found, the program is flagged.
  \item OptiFuzz then fuzzes the flagged programs with various random inputs to test whether the branching instructions can be exploited to leak information about the input.
  \item At last, OptiFuzz reports the results of the fuzzing in the form of a PDF report.
\end{itemize}
The OptiFuzz pipeline is illustrated in Figure \ref{fig:optifuzz-pipeline}. 
Each of the steps in the pipeline is described in detail in the following sections.
\begin{figure}[H]
  \centering
  \input{tikz/optifuzz-pipeline.tex}
  \caption{The OptiFuzz pipeline.}
  \label{fig:optifuzz-pipeline}
\end{figure}

\subsection{Code Generation}
The first step in the OptiFuzz pipeline is code generation. 
The code generation module is written in OCaml and works by generating abstract syntax trees according to the grammar in Figure \ref{fig:grammar}.

\begin{figure}[H]
  \centering
  \begin{align*}
    e \in Expr ::= & x \mid y &\text{(input variables)}\\
    & n \in \{0, 1\}^{64} &\text{(64-bit integer literals)}\\
    & -e \mid e + e \mid e - e \mid e \times e &\text{(arithmetic operators)}\\
    & \texttt{true} \mid \texttt{false} &\text{(boolean literals)}\\
    & !e &\text{(logical operators)}\\
    & e < e \mid e \leq e \mid e > e \mid e \geq e \mid e = e \mid e \neq e &\text{(comparison operators)}\\
    & e \texttt{\&} e \mid e \texttt{|} e \mid \text{\~{}} e \mid e \text{\^{}} e \mid e \ll e \mid e \gg e &\text{(bitwise operators)}
  \end{align*}
  \caption{The grammar that defines the ASTs generated by the code generation module in the OptiFuzz pipeline.}
  \label{fig:grammar}
\end{figure}

The grammar defines all programs with 2 variables and non-branching arithmetic, logical and comparison operations in C \cite{c-standard}, excluding division (\texttt{/}) and modulus (\texttt{\%}).
The reason for excluding division and modulus is that they may cause division-by-zero errors.
Generated programs are forced to include both input variables, $x$ and $y$.
$x$ and $y$ are the inputs to the program, and we require both to be present since programs with 0 or 1 input variables are trivially constant-time.
Furthermore, Booleans are defined as the integer literals 0 and 1 just like in the C standard library \texttt{stdbool} \cite{c-standard}.
An example of a program generated by the code generation module is shown in Figure \ref{fig:code-gen-example}.

\begin{figure}[H]
  \begin{lstlisting}[style=defstyle,language=C]
#define false 0
#define true 1
int program(int x, int y) { return !(y * (43 * (x != true))); } \end{lstlisting}
  \caption{Example of a program generated by the code generation module.}
  \label{fig:code-gen-example}
\end{figure}

The code generation module works by generating a random distribution that selects different symbols in the grammar with a certain probability.
This ensures that not all generated programs will be uniformly random.
For example, if a generated distribution heavily favors the left-shift operator, then the generated programs will contain a lot of left-shift operations.
This is useful for detecting whether certain types of programs are more likely to be constant-time than others.

The grammar in Figure \ref{fig:grammar} generates programs that are not typical 

\todo{Argue why not a more realistic grammar was chosen}
\todo{Find examples of weird real life programs}
\subsubsection{Limitations}

\subsection{Assembly Inspection}
\label{sec:inspection}

The next step in the OptiFuzz pipeline is assembly inspection.
The assembly inspection module is written in Python and works by inspecting the assembly generated by the compiler across different optimization flags.
The compiled assembly is inspected for conditional branching instructions and flagged if that is the case.
The assembly inspection module is only able to analyze x86-64 assembly.

In x86-64 assembly, \texttt{Jcc} (note, this does not include \texttt{JMP}), \texttt{LOOP} and \texttt{LOOPcc} are the only conditional branching instructions \citep{intel-reference}.
This means that the assembly inspection module only needs to look for these instructions.
Both \texttt{Jcc} and \texttt{LOOPcc} refer to families of instructions where \texttt{cc} is a condition code.
For example, \texttt{JE} (jump if equal) is in the \texttt{Jcc} instruction family. 
We did not include conditional move instructions in the analysis since they are constant-time \citep{cmov-is-constant-time}.

\subsubsection{Limitations}
An obvious limitation of this approach is that it is tailored for x86-64 assembly.
Hence our analysis will not work for other architectures.
Furthermore, the programs that are flagged by the assembly inspection module are not necessarily vulnerable to timing attacks.
Hence, the assembly inspection module overapproximates the set of programs that are vulnerable to timing attacks.
For example, the program in Figure \ref{fig:assembly-inspection-example} is flagged by the assembly inspection module, but it is constant-time since both branches take the same amount of time to execute.

\begin{figure}[]
  \centering
  \begin{lstlisting}[style=defstyle, language={[x86masm]Assembler}, basicstyle=\footnotesize\ttfamily,breaklines=true, xleftmargin=4cm, xrightmargin=4cm]
...
  cmpl    $1, -4(%rbp) ; compare TOS to 1
  je      .L2          ; jump to .L2 if equal
  movl    $43, %eax    ; move 43 into %eax
  jmp     .L3
.L2:
  movl    $0, %eax     ; move 0 into %eax
.L3:
  ret\end{lstlisting}
  \caption{Example of a program that is flagged by the assembly inspection module, but is constant-time.}
  \label{fig:assembly-inspection-example}
\end{figure}
\subsection{Fuzzing}
\todo{Refer to the fact I just wrote that the assembly inspection overapproximates}

\subsubsection{Limitations}
\todo{Underapproximates since fuzzing might not be perfect/noise}
\subsection{Statistical Analysis}
\label{sec:statistical-analysis}
\todo{We set t threshold to 10 like in dudect}
\todo{Analysis based on: https://eprint.iacr.org/2015/536.pdf}
\todo{Remember to say that we only visualize top 95\% percentile since we have some noise. However, importantly we always analyze the whole distribution.}

\section{Experimental Results}
\todo{Note: \url{https://ieeexplore.ieee.org/stamp/stamp.jsp?tp=&arnumber=8077809} says clang had issues with not compiling to cmov back in the day (2017)}
All experiments were carried out on an Intel Core i7-9750H running Manjaro Linux 22.1.2 with kernel version Linux 5.15.112-1.

\subsection{Genereal Optimizations}
\todo{O0, ..., Os}

\subsection{Specific Optimizations}
\todo{Single out specific optimizations}

\subsection{Vulnerable operations}
\todo{Try to identify what operations are vulnerable}


\section{Evaluation}
\todo{Evaluation of OptiFuzz and analysis. Is it accurate?}

\section{Solving the Problem}
From the previous sections, it is clear that modern compilers introduce timing vulnerabilities in constant-time programs.
This is devastating for security since it means that developers cannot reliably write constant-time code.
A solution is required.
In this section, we discuss how the problem can be solved and what the implications different solutions have on security and efficiency.

As shown in Section \ref{sec:gcc-vs-clang}, \texttt{clang} is much more conservative when it comes to inserting conditional branches in constant-time programs.
Instead of inserting variable-time conditional branches, \texttt{clang} uses constant-time conditional moves to implement branching control flow.
From this observation, it might be tempting to conclude that always substituting conditional branches with conditional moves is the solution to the problem.
However, there is more to the story.

\texttt{cmov} instructions severely restrict the out-of-order engine on modern CPUs since \texttt{cmov} instructions increase the data dependency between instructions significantly \citep{intel-optimization-reference}. 
Furthermore, branch predictions are not used for \texttt{cmov} instructions, which can lead to significant performance degradation.
Hence, we have to address this trade-off between security and efficiency.

\subsection{Forcing \texttt{cmov} Instructions When Necessary}
This trade-off between security and efficiency has led multiple researchers to propose solutions that only force \texttt{cmov} instructions to be used on critical data \citep{what-you-c,llvm-issues-blog-post}.
In both studies, the authors propose to add the possibility to specify forced \texttt{cmov} instructions in the source code and restrict the backend of the compiler to not being able to optimize the \texttt{cmov} away for those specific instructions.
This solution has the advantage that it does not affect the performance of non-critical parts of the program.
In fact, \citeauthor{what-you-c} show that their solution only introduces a performance penalty of 1\% and sometimes even increases performance.

However, this solution also has its disadvantages.
First, the solution relies heavily on the developer.
The developer has to know which parts of the program are critical, and also be able to use the language extension correctly.
In turn, it means that security relies on human factors, which are error-prone.
Second, the solution relies on the compiler being configurable to avoid forced \texttt{cmov} instructions being optimized away.
As we discussed in Section \ref{sec:gcc-vs-clang} and \ref{sec:general-optimizations}, \texttt{clang} seems to be quite configurable while \texttt{gcc} does not.
Hence, the solution might not be easily applicable to all compilers.

Another solution is to implement vulnerable parts of the program using a domain-specific language that is designed to be constant-time.
\citeauthor{fact} proposed a domain-specific language, called FaCT, for implementing constant-time cryptographic algorithms \citep{fact}.
FaCT is designed to be interoperable with C, and the compiler ensures constant-time code by leveraging the \texttt{ct-verif} tool that is based on verification of the LLVM IR \cite{verifying-constant-time-llvm}.
\texttt{ct-verif} operates on a formalization of constant-time code and verifies based on a theoretically sound and complete methodology.
However, \citeauthor{verifying-constant-time-llvm} note that the verification process of \texttt{ct-verif} is not guaranteed to identify all timing vulnerabilities due to translation between the LLVM IR and machine code, and due to possible incompleteness of their formalization of constant-time code. 
Some of these issues were experimentally verified.
However, generally \texttt{ct-verif} captures significantly more timing vulnerabilities than other verification alternatives \citep{verifying-constant-time-llvm}.
Additionally, \texttt{ct-verif} is efficient at verifying constant-time code, often being much faster than the compilation step, but occasionally being a few times slower.

\subsection{Bullet-proofing the Compiler}
A more radical approach is to force the compiler to always produce constant-time code.
\citeauthor{verified-constant-time-c-comiler} provides a formally verified constant-time C compiler, built on top of the CompCert verified C compiler \citep{verified-constant-time-c-comiler}. 
The constant-time CompCert compiler guarantees that constant-time source code is compiled into constant-time machine code.
This is achieved by instrumenting the operational semantics of CompCert's IR to be able to capture cryptographic constant-time and proving that cryptographic constant-time is preserved during compilation. 
This has immense advantages for security.

However, the security advantages come at the cost of efficiency. 
The constant-time CompCert compiler is significantly slower than \texttt{gcc}.
Experimental results show an efficiency penalty of well over 100\% in some cases \citep{verified-constant-time-c-comiler}.

\subsection{Testing Solutions}

\todo{Black-box solutions. Argue that it is too vague, but perhaps OptiFuzz can be used to minimize issues from compilers.}

\section{Conclusion and Future Work}
\todo{Future Work: More complicated programs}

\pagebreak

\bibliographystyle{abbrvnat}
\bibliography{litterature}

\pagebreak
\begin{appendices}

\section{Select General Optimizations Results}
\label{appendix:general-optimizations-results}
%\documentclass[10pt]{article}
\input{prelude}
\usepackage[authoryear]{natbib}

\title{
  Optimizing Away Security in C
}

\groupNumber{10}
\author{Anders B. Clausen \and Johan T. Degn \and Jonathan S. Eilath}

\begin{document}
\maketitle
\thispagestyle{empty}

\todo{Add short guide to read data plots}
\todo{We can probably add "what compile flags are dangerous" to contributions}
\todo{Make figures smaller and better, and remove [H]}

\section*{Abstract}
\todo{After the paper is done}
\todo{Write something like: the status quo is the worst of both worlds}

\section{Introduction}
\input{chapters/introduction.tex}

\section{Preliminaries}
\input{chapters/preliminaries.tex}

\section{OptiFuzz}
\input{chapters/optifuzz.tex}
\subsection{Code Generation}
\input{chapters/code-generation.tex}
\subsection{Assembly Inspection}
\input{chapters/assembly-inspection.tex}
\subsection{Fuzzing}
\input{chapters/fuzzing.tex}
\subsection{Statistical Analysis}
\input{chapters/statistical-analysis.tex}

\section{Experimental Results}
\input{chapters/results.tex}

\section{Evaluation}
\input{chapters/evaluation.tex}

\section{Solving the Problem}
\input{chapters/discussion.tex}

\section{Conclusion and Future Work}
\input{chapters/conclusion.tex}

\pagebreak

\bibliographystyle{abbrvnat}
\bibliography{litterature}

\pagebreak
\begin{appendices}

\section{Select General Optimizations Results}
\label{appendix:general-optimizations-results}
%\input{appendix/general-optimizations-results/master.tex}

\end{appendices}

\end{document}

\end{appendices}

\end{document}

\end{appendices}

\end{document}

\end{appendices}

\end{document}