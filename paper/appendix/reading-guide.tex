\section*{Appendix Reading Guide}
The following is a guide on how to interpret the results produced by OptiFuzz. 

At the top of a page, a program signature line will be shown:
\begin{center}
    \textbf{Program 1} -- \texttt{Seed 257634744} -- \texttt{gcc O0 O1 O2 O3}\\\small\texttt{Classes: uniform fixed xzero yzero xlty yltx small equal max64}\\[2mm]
\end{center}
The seed used to generate the program is printed, followed by the compiler and compiler flags.
On a new line, all input classes that the fuzzer used are printed.
Refer to \ref{sec:fuzzing} for a description of each input class.

Next, the program is shown:
\begin{lstlisting}[style=defstyle,language=C]
...
int program(long long int x, long long int y) { return (((44 ^ ((false >> (-874552120269441001 == x)) | ((53 * 29) >= (-5509653365268467218 ^ x)))) * 24) & (false + x)); }
\end{lstlisting}
The first two lines of the program have been trimmed. Namely, two define statements
which define \texttt{false} to 0 and \texttt{true} to 1.
What follows now is a histogram of the measured times:
\begin{figure}[H]
    \centering
\begin{subfigure}[T]{0.30333333333333334\textwidth}
\caption*{O0}
\begin{lrbox}{\mybox}%
\begin{tikzpicture}[>=latex]
            \begin{axis}[
                axis x line=center,
                axis y line=center,
                name=ax,
                scaled y ticks=base 10:-3,
                ytick scale label code/.code={},
                yticklabel={\pgfmathprintnumber{\tick} k},
                xlabel={CPU Clocks},
                ylabel={Frequency},
                x label style={at={(axis description cs:0.5,-0.1)},anchor=north},
                y label style={at={(axis description cs:-0.1,.5)},rotate=90,anchor=south,yshift=4mm},
                area style,
                ymin=0,
                xmin=17,
                xmax=35,
                ymax=1205
                ]
                \addplot+ [ybar interval,mark=no,color=firstCol,fill=firstCol,fill opacity=0.2] table {
                    18 238
19 380
20 30
21 44
22 17
23 9
24 21
25 15
26 2
27 8
28 24
29 89
30 133
31 64
32 0
33 0
34 3
                };
\addplot+ [ybar interval,mark=no,color=firstCol,fill=firstCol,fill opacity=0.2] table {
                    17 1121
18 2
19 2
                };
\addplot+ [ybar interval,mark=no,color=firstCol,fill=firstCol,fill opacity=0.2] table {
                    17 317
18 538
19 37
20 16
21 14
22 8
23 10
24 106
25 24
26 1
27 0
28 0
29 1
30 0
31 0
32 0
33 3
                };
\addplot+ [ybar interval,mark=no,color=firstCol,fill=firstCol,fill opacity=0.2] table {
                    18 1148
19 8
                };
\addplot+ [ybar interval,mark=no,color=firstCol,fill=firstCol,fill opacity=0.2] table {
                    17 228
18 401
19 56
20 23
21 38
22 18
23 14
24 250
25 46
26 0
27 0
28 1
29 2
30 1
31 0
32 0
33 1
                };
\addplot+ [ybar interval,mark=no,color=firstCol,fill=firstCol,fill opacity=0.2] table {
                    17 140
18 288
19 92
20 40
21 58
22 23
23 28
24 357
25 88
26 0
27 0
28 1
29 1
30 0
31 0
32 0
33 1
34 1
                };
\addplot+ [ybar interval,mark=no,color=firstCol,fill=firstCol,fill opacity=0.2] table {
                    17 1130
18 3
19 3
                };
\addplot+ [ybar interval,mark=no,color=firstCol,fill=firstCol,fill opacity=0.2] table {
                    17 15
18 858
19 23
20 28
21 6
22 9
23 18
24 2
25 6
26 57
27 91
28 3
29 0
30 1
31 1
32 0
33 1
                };
\addplot+ [ybar interval,mark=no,color=firstCol,fill=firstCol,fill opacity=0.2] table {
                    17 246
18 418
19 51
20 19
21 22
22 25
23 17
24 112
25 201
26 1
27 1
28 0
29 1
30 0
31 0
32 0
33 0
34 1
                };
                                    \draw[color=black, line width=0.2mm, dashed] 
                    (axis cs:22, -60.25) -- (axis cs:22, 1205);
                    \draw[color=black, line width=0.2mm, dashed] 
                    (axis cs:17, -60.25) -- (axis cs:17, 1205);
                    \draw[color=black, line width=0.2mm, dashed] 
                    (axis cs:19, -60.25) -- (axis cs:19, 1205);
                    \draw[color=black, line width=0.2mm, dashed] 
                    (axis cs:18, -60.25) -- (axis cs:18, 1205);
                    \draw[color=black, line width=0.2mm, dashed] 
                    (axis cs:20, -60.25) -- (axis cs:20, 1205);
                    \draw[color=black, line width=0.2mm, dashed] 
                    (axis cs:21, -60.25) -- (axis cs:21, 1205);
                    \draw[color=black, line width=0.2mm, dashed] 
                    (axis cs:17, -60.25) -- (axis cs:17, 1205);
                    \draw[color=black, line width=0.2mm, dashed] 
                    (axis cs:19, -60.25) -- (axis cs:19, 1205);
                    \draw[color=black, line width=0.2mm, dashed] 
                    (axis cs:20, -60.25) -- (axis cs:20, 1205);
            \end{axis} 
            \node[below=15mm of ax] (1) {
                $\begin{aligned}
                    \texttt{equal}_\mu: & \,22\\
                    \texttt{fixed}_\mu: & \,17\\
                    \texttt{max64}_\mu: & \,19
                \end{aligned}$
            };
            \node[left=4mm of 1] (2) {
                $\begin{aligned}
                    \texttt{small}_\mu: & \,18\\
                    \texttt{uniform}_\mu: & \,20\\
                    \texttt{xlty}_\mu: & \,21
                \end{aligned}$
            };
            \node[right=4mm of 1] (3) {
                $\begin{aligned}
                    \texttt{xzero}_\mu: & \,17\\
                    \texttt{yltx}_\mu: & \,19\\
                    \texttt{yzero}_\mu: & \,20
                \end{aligned}$
            };
            \node[fit=(1)(2)(3),draw]{};
            \end{tikzpicture}%
        \end{lrbox}\resizebox{\textwidth}{!}{\usebox{\mybox}}
\end{subfigure}
\end{figure}
The histogram caption at the top indicates which compiler flag was used to compile the program.
The table below the histogram indicates the means for the different input classes.
These means have been drawn as dashed lines over the histogram.

The histogram plot is not conventional, as the saturation indicates agreement across all input classes.
That is, if a bucket is close to transparent, then that might indicate, that it only came from one input class.
If multiple input classes have bins at the same location with the same height, the bin will be very saturated. 
If we examine the histogram above, we see that some input classes got about a total of about 1000 measurements in the [25, 32] interval.
As no mean sticks out - other than the \texttt{equal} class - it might be disregarded as noise.

Next is the assembly, hypothesis rejection flag, and jump indicators, which should be directly below the histogram.
\begin{figure}[H]
    \centering
\begin{subfigure}[T]{0.2733333333333333\textwidth}
\vspace*{2mm}\tiny {\color{red}$H_0$ REJECTED!}\ \vspace*{2mm}\tiny jg [10]
\begin{lstlisting}[style=defstyle,language={[x86masm]Assembler},basicstyle=\tiny\ttfamily,breaklines=true]
...
	endbr64
	pushq	%rbp
	movq	%rsp, %rbp
	movq	%rdi, -8(%rbp)
	movq	%rsi, -16(%rbp)
	movabsq	$-5509653365268467218, %rax
	xorq	-8(%rbp), %rax
	cmpq	$1537, %rax
	jg	.L2
	movl	$1080, %edx
	jmp	.L3
.L2:
	movl	$1056, %edx
.L3:
	movq	-8(%rbp), %rax
	andl	%edx, %eax
	popq	%rbp
	ret
...\end{lstlisting}
\end{subfigure}
\end{figure}
Only if the hypothesis was rejected, it will be marked with red.

The jump indicator \texttt{jg [10]} tells us, that a \texttt{jg} instruction was found on line 10.

\newpage