\textbf{Program 2} -- \texttt{Seed 104748697}

\begin{lstlisting}[style=defstyle, language=C]
...
int program(int x, int y) { return (((y << 1) >= y) >= (true + (x == y))); } 
\end{lstlisting}


\begin{figure}[H]
\begin{subfigure}[T]{.3\textwidth}
\caption*{GCC O0}
\begin{lrbox}{\mybox}%
\begin{tikzpicture}[>=latex]
    \begin{axis}[
    axis x line=center,
    axis y line=center,
    scaled y ticks=base 10:-3,
    ytick scale label code/.code={},
    yticklabel={\pgfmathprintnumber{\tick} k},
    xlabel={Clock Count},
    ylabel={Frequency},
    x label style={at={(axis description cs:0.5,-0.1)},anchor=north},
    y label style={at={(axis description cs:-0.1,.5)},rotate=90,anchor=south,yshift=4mm},
    area style,
    ymin= 0,
    xmin= 272, % minus 10
    xmax= 532, % plus 10
    ymax= 120000] % round 5000 up
    \addplot+ [ybar interval,mark=no,
    color=firstCol,
    fill=firstCol, 
    fill opacity=0.5] table {
        282 4913
        284 18009
        287 17054
        289 235
        292 26215
        294 72201
        296 47800
        299 5334
        301 13594
        304 115949
        306 71909
        309 9463
        311 3259
        313 47945
        316 115409
        318 29307
        321 1392
        323 6003
        325 88804
        328 100972
        330 23288
        333 417
        335 3915
        338 60942
        340 71524
        342 13644
        345 702
        347 72
        350 3484
        352 11305
        355 8850
        357 261
        359 2
        362 3
        364 101
        367 1317
        369 930
        371 244
        374 22
        376 3
        379 2
        381 13
        384 13
        386 2
        388 1
        391 11
        393 24
        396 27
        398 0
        401 0
        403 1
        405 1
        408 32
        410 9
        413 2
        415 0
        417 0
        420 13
        422 60
        425 296
        427 440
        430 270
        432 2
        434 2
        437 5
        439 4
        442 68
        444 149
        446 114
        449 83
        451 53
        454 109
        456 83
        459 152
        461 111
        463 71
        466 115
        468 82
        471 135
        473 60
        476 41
        478 57
        480 45
        483 100
        485 47
        488 63
        490 43
        492 47
        495 60
        497 19
        500 13
        502 10
        505 12
        507 8
        509 8
        512 5
        514 8
        517 1
        519 3
        522 2
    };
    \end{axis}
\end{tikzpicture}%
\end{lrbox}\resizebox{\textwidth}{!}{\usebox{\mybox}}
\end{subfigure}%
\begin{subfigure}[T]{.3\textwidth}
\caption*{GCC O2}
\begin{lrbox}{\mybox}%
\begin{tikzpicture}[>=latex]
    \centering
    \begin{axis}[
    axis x line=center,
    axis y line=center,
    scaled y ticks=base 10:-3,
    ytick scale label code/.code={},
    yticklabel={\pgfmathprintnumber{\tick} k},
    xlabel={Clock Count},
    ylabel={Frequency},
    x label style={at={(axis description cs:0.5,-0.1)},anchor=north},
    y label style={at={(axis description cs:-0.1,.5)},rotate=90,anchor=south,yshift=4mm},
    xmin= 272, % minus 10
    xmax= 532, % plus 10
    ymin= 0,
    ymax= 120000, % round 5000 up
    area style] 
    \addplot+ [ybar interval,mark=no,
    color=secondCol,
    fill=secondCol, 
    fill opacity=0.5] table {
        282 4913
        284 18009
        287 17054
        289 235
        292 26215
        294 72201
        296 47800
        299 5334
        301 13594
        304 115949
        306 71909
        309 9463
        311 3259
        313 47945
        316 115409
        318 29307
        321 1392
        323 6003
        325 88804
        328 100972
        330 23288
        333 417
        335 3915
        338 60942
        340 71524
        342 13644
        345 702
        347 72
        350 3484
        352 11305
        355 8850
        357 261
        359 2
        362 3
        364 101
        367 1317
        369 930
        371 244
        374 22
        376 3
        379 2
        381 13
        384 13
        386 2
        388 1
        391 11
        393 24
        396 27
        398 0
        401 0
        403 1
        405 1
        408 32
        410 9
        413 2
        415 0
        417 0
        420 13
        422 60
        425 296
        427 440
        430 270
        432 2
        434 2
        437 5
        439 4
        442 68
        444 149
        446 114
        449 83
        451 53
        454 109
        456 83
        459 152
        461 111
        463 71
        466 115
        468 82
        471 135
        473 60
        476 41
        478 57
        480 45
        483 100
        485 47
        488 63
        490 43
        492 47
        495 60
        497 19
        500 13
        502 10
        505 12
        507 8
        509 8
        512 5
        514 8
        517 1
        519 3
        522 2
    };
    \end{axis}
\end{tikzpicture}%
\end{lrbox}
\resizebox{\textwidth}{!}{\usebox{\mybox}}
\end{subfigure}%
\begin{subfigure}[T]{.3\textwidth}
\caption*{GCC O3}
\begin{lrbox}{\mybox}%
\begin{tikzpicture}[>=latex]
    \centering
    \begin{axis}[
    axis x line=center,
    axis y line=center,
    scaled y ticks=base 10:-3,
    ytick scale label code/.code={},
    yticklabel={\pgfmathprintnumber{\tick} k},
    xlabel={Clock Count},
    ylabel={Frequency},
    x label style={at={(axis description cs:0.5,-0.1)},anchor=north},
    y label style={at={(axis description cs:-0.1,.5)},rotate=90,anchor=south,yshift=4mm},
    xmin= 272, % minus 10
    xmax= 532, % plus 10
    ymin= 0,
    ymax= 120000, % round 5000 up
    area style] 
    \addplot+ [ybar interval,mark=no,
    color=thirdCol,
    fill=thirdCol, 
    fill opacity=0.5] table {
        282 4913
        284 18009
        287 17054
        289 235
        292 26215
        294 72201
        296 47800
        299 5334
        301 13594
        304 115949
        306 71909
        309 9463
        311 3259
        313 47945
        316 115409
        318 29307
        321 1392
        323 6003
        325 88804
        328 100972
        330 23288
        333 417
        335 3915
        338 60942
        340 71524
        342 13644
        345 702
        347 72
        350 3484
        352 11305
        355 8850
        357 261
        359 2
        362 3
        364 101
        367 1317
        369 930
        371 244
        374 22
        376 3
        379 2
        381 13
        384 13
        386 2
        388 1
        391 11
        393 24
        396 27
        398 0
        401 0
        403 1
        405 1
        408 32
        410 9
        413 2
        415 0
        417 0
        420 13
        422 60
        425 296
        427 440
        430 270
        432 2
        434 2
        437 5
        439 4
        442 68
        444 149
        446 114
        449 83
        451 53
        454 109
        456 83
        459 152
        461 111
        463 71
        466 115
        468 82
        471 135
        473 60
        476 41
        478 57
        480 45
        483 100
        485 47
        488 63
        490 43
        492 47
        495 60
        497 19
        500 13
        502 10
        505 12
        507 8
        509 8
        512 5
        514 8
        517 1
        519 3
        522 2
    };
    \end{axis}
\end{tikzpicture}%
\end{lrbox}
\resizebox{\textwidth}{!}{\usebox{\mybox}}
\end{subfigure}%
\end{figure}
\vspace*{-6mm} % Hack to move figures closer, in order to squeeze a bit more space for asm
\begin{figure}[H]
\centering
\begin{subfigure}[T]{.3\textwidth}
\begin{lstlisting}[style=defstyle, language={[x86masm]Assembler},basicstyle=\tiny\ttfamily,
    framexrightmargin=-8mm]
    program:
.LFB0:
	.cfi_startproc
	endbr64
	xorl	%eax, %eax
	cmpl	$71, %esi
	movl	%esi, %ecx
	movl	%esi, %edi
	setg	%al
	movl	%esi, %edx
	sall	$27, %edi
	sall	%cl, %eax
	cmpl	$94, %esi
	movl	$94, %esi
	movl	%edi, %r9d
	setg	%r8b
	sall	%cl, %esi
	negl	%r9d
	movl	%edx, %r10d
	testl	%esi, %esi
	notl	%r10d
	setle	%sil
	movzbl	%sil, %esi
	sall	%cl, %esi
	addl	%eax, %esi
	setne	%sil
	subl	$1, %edi
	cmpl	%r9d, %eax
	setne	%cl
	movzbl	%sil, %esi
	cmpb	%cl, %r8b
	notl	%esi
	seta	%cl
	movzbl	%cl, %ecx
	cmpl	%ecx, %edi
	setle	%cl
	movzbl	%cl, %ecx
	andl	%ecx, %esi
	sall	%cl, %esi
	addl	%eax, %esi
	setne	%sil
	subl	$1, %edi
	cmpl	%r9d, %eax
	setne	%cl
	movzbl	%sil, %esi
	cmpb	%cl, %r8b
	notl	%esi
	seta	%cl
	movzbl	%cl, %ecx
	cmpl	%ecx, %edi
	setle	%cl
	movzbl	%cl, %ecx
	andl	%ecx, %esi
	sall	%cl, %esi
	addl	%eax, %esi
	setne	%sil
	subl	$1, %edi
	cmpl	%r9d, %eax
	setne	%cl
	movzbl	%sil, %esi
	cmpb	%cl, %r8b
	notl	%esi
	seta	%cl
	movzbl	%cl, %ecx
	cmpl	%ecx, %edi
	setle	%cl
	movzbl	%cl, %ecx
	andl	%ecx, %esi
	movzbl	%cl, %ecx
	andl	%ecx, %esi
	movzbl	%cl, %ecx
	andl	%ecx, %esi
	movzbl	%cl, %ecx
	andl	%ecx, %esi
	movzbl	%cl, %ecx
\end{lstlisting}
\end{subfigure}%
\begin{subfigure}[T]{.3\textwidth}
\begin{lstlisting}[style=defstyle, language={[x86masm]Assembler},basicstyle=\tiny\ttfamily,
    framexrightmargin=-8mm]
    program:
.LFB0:
	.cfi_startproc
	endbr64
	xorl	%eax, %eax
	cmpl	$71, %esi
	movl	%esi, %ecx
	movl	%esi, %edi
	setg	%al
	movl	%esi, %edx
	sall	$27, %edi
	sall	%cl, %eax
	cmpl	$94, %esi
	movl	$94, %esi
	movl	%edi, %r9d
	setg	%r8b
	sall	%cl, %esi
	negl	%r9d
	movl	%edx, %r10d
	testl	%esi, %esi
	notl	%r10d
	setle	%sil
	movzbl	%sil, %esi
	sall	%cl, %esi
	addl	%eax, %esi
	setne	%sil
	subl	$1, %edi
	cmpl	%r9d, %eax
	setne	%cl
	movzbl	%sil, %esi
	cmpb	%cl, %r8b
	notl	%esi
	seta	%cl
	movzbl	%cl, %ecx
	cmpl	%ecx, %edi
	setle	%cl
	movzbl	%cl, %ecx
	andl	%ecx, %esi
\end{lstlisting}
\end{subfigure}%
\begin{subfigure}[T]{.3\textwidth}
\begin{lstlisting}[style=defstyle, 
    language={[x86masm]Assembler},
    basicstyle=\tiny\ttfamily,
    framexrightmargin=-8mm]
    program:
.LFB0:
	.cfi_startproc
	endbr64
	xorl	%eax, %eax
	cmpl	$71, %esi
	movl	%esi, %ecx
	movl	%esi, %edi
	setg	%al
	movl	%esi, %edx
	sall	$27, %edi
	sall	%cl, %eax
	cmpl	$94, %esi
	movl	$94, %esi
	movl	%edi, %r9d
	setg	%r8b
	sall	%cl, %esi
	negl	%r9d
	movl	%edx, %r10d
	testl	%esi, %esi
	notl	%r10d
	setle	%sil
	movzbl	%sil, %esi
	sall	%cl, %esi
\end{lstlisting}
\end{subfigure}%
\end{figure}


\newpage































