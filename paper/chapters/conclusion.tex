
We have provided a tool, OptiFuzz, that can be used to test compilers for timing vulnerabilities.
OptiFuzz has been optimized considering the C language specification, the x86-64 architecture, and state-of-the-art statistical analysis.
This has resulted in a simple tool that is indicative of the real-world issue of timing vulnerabilities introduced by compilers.
With OptiFuzz, we have shown that between 0.15\% and 12.63\% of randomly generated programs are vulnerable to timing attacks when compiled with \texttt{gcc} using various optimization flags.
We have used our results to pinpoint specific optimizations within \texttt{gcc} that are responsible for introducing timing vulnerabilities, as well as specific expression patterns that are vulnerable to resulting in timing vulnerabilities when compiled.
We have also shown that due to better utilization of \texttt{cmovcc} instructions, \texttt{clang} introduces significantly fewer timing vulnerabilities -- so much that we could not detect any timing vulnerabilities introduced by \texttt{clang} using our methodology.
In light of this result, we have discussed how the issue of timing vulnerabilities introduced by compilers can be mitigated by using language-based security techniques from the literature.
Generally, solutions to this problem represent a trade-off between security and performance where very secure solutions are not that performant and vice versa.

In the light of our research, we have identified several areas for future work:
Extend OptiFuzz to generate more complicated programs, eg. by reintroducing the division and modulus operators while dealing with division by zero in a meaningful way.
Implement other relevant input classes. Perhaps a dynamic class containing the hardcoded values from a program. This might be able to catch the very specific but unpredictable values that cause branching.
Extend the statistical analysis and consider results from all input classes. This could include researching the usability of other t-tests.
Create a tool that warns developers in real-time about constant C expressions that might be compiled to variable-time by comparing it to a database precomputed using OptiFuzz. This tool could furthermore be implemented in a CI/CD environment.
Extend OptiFuzz to support other processor architectures, especially ones used by embedded systems.
