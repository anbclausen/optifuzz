We have run OptiFuzz to generate around 600,000 programs, which have been fuzzed and analyzed under different compilers and optimization flags.
In this section, we present our results.

We have divided this section into four sections.
First, we compare \texttt{gcc} and \texttt{clang} generally in terms of the number of timing vulnerabilities introduced.
Second, we present our results for general optimizations flags such as \texttt{O2} and \texttt{Os}.
Third, we present our results regarding specific optimizations.
Finally, we present our results regarding what operations are most likely to cause timing vulnerabilities when optimized.
All experiments were carried out on an Intel Core i7-9750H running Manjaro Linux 22.1.2 with kernel version Linux 5.15.112-1.

\subsection{gcc vs. clang}
\todo{Note: \url{https://ieeexplore.ieee.org/stamp/stamp.jsp?tp=&arnumber=8077809} says clang had issues with not compiling to cmov back in the day (2017)}

\subsection{Genereal Optimizations}
\todo{O0, ..., Os}

\subsection{Specific Optimizations}
\todo{Single out specific optimizations}

\subsection{Vulnerable operations}
\todo{Try to identify what operations are vulnerable}
