The C programming language is one of the most widely used programming languages in the world. 
It is used in a wide variety of applications, ranging from embedded systems to cryptographic libraries. 
However, C is also notorious for lacking security guarantees.
Security-related issues in programs written in C stem from both the programmer, but also from the compiler.

The problem with security issues introduced by the compiler is especially prevalent in the context of timing attacks on cryptographic algorithms. 
Developers try to mitigate timing attacks by writing constant-time code -- code where the execution time is independent of the input.
However, the compiler may introduce timing leaks by optimizing away the constant-time code.
In a recent study \citep{developer-survey-timing-attacks}, it was shown that the vast majority of developers of cryptographic libraries rely on constant-time code practices that in theory result in constant-time code, but may be vulnerable after compilation.

The issue of timing vulnerabilities introduced by compilers is well-known in the community.
Several proposals have been made to mitigate the issue \citep{what-you-c, dudect, fact, verified-constant-time-c-comiler}.
However, the problem persists and to the best of our knowledge, no study has quantified the issue of timing vulnerabilities introduced by compilers.

In this paper, we try to quantify the issue of timing vulnerabilities introduced by the two most popular C compilers, gcc\footnote{\url{https://gcc.gnu.org/}.} and clang\footnote{\url{https://clang.llvm.org/}.}.
We do this by implementing a tool, OptiFuzz, that generates, analyzes and fuzzes random C programs.
For simplicity, the generated programs are limited to only containing non-branching arithmetic, logical and comparison operations.
Also \texttt{/} and \texttt{\%} are avoided, to mitigate division-by-zero errors.
We investigate what optimization flags are responsible for introducing timing leaks and compare them.
Finally, we discuss how the issue of timing vulnerabilities introduced by the compiler can be mitigated by using language-based security techniques.

\subsection{Related Work}
The issue of timing vulnerabilities introduced by the clang C compiler across different versions has been researched by Simon et. al. \citep{what-you-c}. 
Several studies have investigated potential solutions to the issue, including using constant-time branching instructions \citep{what-you-c}, using black-box testing software \citep{dudect}, domain-specific languages \citep{fact}, and notably a verified constant-time C compiler has been developed \citep{verified-constant-time-c-comiler}.

\subsection{Contributions}
We provide a quantitative analysis of timing vulnerabilities introduced by gcc and clang, focusing on specific troublesome optimization flags.
Additionally, we provide a tool, OptiFuzz, that can be used to generate, analyze and fuzz random C programs for further investigation of the issue.
Finally, we provide a discussion of how the issue of timing vulnerabilities introduced by the compiler can be mitigated by using language-based security techniques found in other literature.
\todo{Might be more when we finish the paper}
\subsection{Overview}
This paper is organized as follows: $\ldots$ 
\todo{After we have finished the rest of the paper}
\todo{Insert guide on how to read appendix here??}